\begin{appendices}
%\appendixpage
\chapter{Supplemental materials}
A data storage medium containing all raw data and necessary analyses to reproduce the results presented in the main body of the dissertation is attached. \\
On the storage, there are three main directories. One contains the \textsc{Madingley Model} (Folder: "Model")in its version used for the simulations here, one all of the analyses (Folder: "Analyses"), and one contains all supplemental material from \cite{Harfoot2014} for the analyses as well as the model parameters used for the simulations (Folder: "Supplemental Material").


\chapter{Figures}
\begin{figure}
\centering
\includestandalone[width=\textwidth]{res/fig/BMD_ts_initial}
\caption[Initial carnivore biomass density for the aseasonal system]{Initical carnivore median biomass density (top), and trend values, i.e. after random and periodic dynamics have been removed, (bottom) for the aseasonal system (exemplary). Note, that \textit{exp. 8} is not displayed due to zero values. Seeding biomass density may have had decisive effect for equilibrium conditions.}
\label{fig:app:tsinit}
\end{figure}
%%%%
%%%
\begin{figure}[h!]
\centering
\includestandalone[width=\textwidth]{res/fig/ts_expno1}
\caption[Log-body mass density time series for \textit{exp. 1} in both systems]{Development of biomass density for the aseasonal (top) and seasonal (bottom) during 100 simulated years for autotrophs (green), herbivores (blue), omnivores (yellow) and carnivores (red) in \textit{exp. 1}. Lines are medians from 100 simulations and shaded areas represent the 95~\% confidence interval from bootstrapping. Both systems differ systematically regarding trophic level biomass dynamics}
\label{fig:chap:res:ts:expno1}
\end{figure}

\clearpage

%%%
\begin{figure}[h!]
\centering
\includestandalone[width=\textwidth]{res/fig/ts_expno2}
\caption[Log-body mass density time series for \textit{exp. 2} in both systems]{Development of biomass density for the aseasonal (top) and seasonal (bottom) during 100 simulated years for autotrophs (green), herbivores (blue), omnivores (yellow) and carnivores (red) in \textit{exp. 2}. Lines are medians from 100 simulations and shaded areas represent the 95~\% confidence interval from bootstrapping. Both systems differ systematically regarding trophic level biomass dynamics}
\label{fig:chap:res:ts:expno2}
\end{figure}

\clearpage


%%%
\begin{figure}[h!]
\centering
\includestandalone[width=\textwidth]{res/fig/ts_expno3}
\caption[Log-body mass density time series for \textit{exp. 3} in both systems]{Development of biomass density for the aseasonal (top) and seasonal (bottom) during 100 simulated years for autotrophs (green), herbivores (blue), omnivores (yellow) and carnivores (red) in \textit{exp. 3}. Lines are medians from 100 simulations and shaded areas represent the 95~\% confidence interval from bootstrapping. Both systems differ systematically regarding trophic level biomass dynamics}
\label{fig:chap:res:ts:expno3}
\end{figure}

\clearpage
%%%
%%

%%%
\begin{figure}[h!]
\centering
\includestandalone[width=\textwidth]{res/fig/ts_expno4}
\caption[Log-body mass density time series for \textit{exp. 4} in both systems]{Development of biomass density for the aseasonal (top) and seasonal (bottom) during 100 simulated years for autotrophs (green), herbivores (blue), omnivores (yellow) and carnivores (red) in \textit{exp. 4}. Lines are medians from 100 simulations and shaded areas represent the 95~\% confidence interval from bootstrapping. Both systems differ systematically regarding trophic level biomass dynamics}
\label{fig:chap:res:ts:expno4}
\end{figure}

\clearpage


%%%
\begin{figure}[h!]
\centering
\includestandalone[width=\textwidth]{res/fig/ts_expno5}
\caption[Log-body mass density time series for \textit{exp. 5} in both systems]{Development of biomass density for the aseasonal (top) and seasonal (bottom) during 100 simulated years for autotrophs (green), herbivores (blue), omnivores (yellow) and carnivores (red) in \textit{exp. 5}. Lines are medians from 100 simulations and shaded areas represent the 95~\% confidence interval from bootstrapping. Both systems differ systematically regarding trophic level biomass dynamics}
\label{fig:chap:res:ts:expno5}
\end{figure}

\clearpage


%%%
\begin{figure}[h!]
\centering
\includestandalone[width=\textwidth]{res/fig/ts_expno6}
\caption[Log-body mass density time series for \textit{exp. 6} in both systems]{Development of biomass density for the aseasonal (top) and seasonal (bottom) during 100 simulated years for autotrophs (green), herbivores (blue), omnivores (yellow) and carnivores (red) in \textit{exp. 6}. Lines are medians from 100 simulations and shaded areas represent the 95~\% confidence interval from bootstrapping. Both systems differ systematically regarding trophic level biomass dynamics}
\label{fig:chap:res:ts:expno6}
\end{figure}

\clearpage


%%%
\begin{figure}[h!]
\centering
\includestandalone[width=\textwidth]{res/fig/ts_expno7}
\caption[Log-body mass density time series for \textit{exp. 7} in both systems]{Development of biomass density for the aseasonal (top) and seasonal (bottom) during 100 simulated years for autotrophs (green), herbivores (blue), omnivores (yellow) and carnivores (red) in \textit{exp. 7}. Lines are medians from 100 simulations and shaded areas represent the 95~\% confidence interval from bootstrapping. Both systems differ systematically regarding trophic level biomass dynamics. Note, that here the data was transformed ($log_{10}(x+1)$).}
\label{fig:chap:res:ts:expno7}
\end{figure}

\clearpage


%%%
\begin{figure}[h!]
\centering
\includestandalone[width=\textwidth]{res/fig/ts_expno8}
\caption[Log-body mass density time series for \textit{exp. 8} in both systems]{Development of biomass density for the aseasonal (top) and seasonal (bottom) during 100 simulated years for autotrophs (green), herbivores (blue), omnivores (yellow) and carnivores (red) in \textit{exp. 8}. Lines are medians from 100 simulations and shaded areas represent the 95~\% confidence interval from bootstrapping. Both systems differ systematically regarding trophic level biomass dynamics}
\label{fig:chap:res:ts:expno8}
\end{figure}

\clearpage

\chapter{Tables}
%% latex table generated in R 3.1.3 by xtable 1.7-4 package
% Thu Jul 23 14:28:11 2015
\begin{table}[ht]
\centering
\caption{Initial ($t = 1$) biomass density [$kg\cdot km^{-2}$] in the aseasonal (top) and seasonal (bottom) system for all experiments.} 
\label{tab:app:initialBMD}
\begin{tabular*}{\textwidth}{@{\extracolsep{\fill} }ccccc}
  \toprule
\textbf{Experiment} & \textbf{autotroph} & \textbf{herbivore} & \textbf{omnivore} & \textbf{carnivore} \\ 
  \midrule
1 & 3497787.37 & 242261.43 & 235485.41 & 213827.12 \\ 
  2 & 3497787.37 & 271421.40 & 264829.44 & 204436.57 \\ 
  3 & 3497787.37 & 271969.08 & 268020.89 & 78013.86 \\ 
  4 & 3497787.37 & 268435.81 & 267075.10 & 201853.16 \\ 
  5 & 3497787.37 & 310664.77 & 302392.36 & 44902.21 \\ 
  6 & 3497787.37 & 310836.82 & 306179.09 & 186371.46 \\ 
  7 & 3497787.37 & 310909.18 & 303400.83 & 44105.21 \\ 
  8 & 3497787.37 & 359780 & 356951.20 &  \\ 
  \cmidrule(lr){2-5}
   1 & 2399780.39 & 242127.95 & 236596.79 & 213827.12 \\ 
  2 & 2399780.39 & 271476.80 & 264984.12 & 203833.68 \\ 
  3 & 2399780.39 & 271969.08 & 268020.89 & 78013.86 \\ 
  4 & 2399780.39 & 268429.88 & 267075.10 & 202602.22 \\ 
  5 & 2399780.39 & 310057.76 & 302549.55 & 44883.61 \\ 
  6 & 2399780.39 & 311216.47 & 306451.11 & 187361.26 \\ 
  7 & 2399780.39 & 309274.79 & 302839.65 & 44211.79 \\ 
  8 & 2399780.39 & 360325.51 & 357285.42 &  \\ 
   \bottomrule
\end{tabular*}
\end{table}

%% latex table generated in R 3.1.3 by xtable 1.7-4 package
% Thu Jul 23 14:28:11 2015
\begin{table}[ht]
\centering
\caption{Initial ($t = 1$) biomass density [$kg\cdot km^{-2}$] in the aseasonal (top) and seasonal (bottom) system for all experiments.} 
\label{tab:app:initialBMD}
\begin{tabular*}{\textwidth}{@{\extracolsep{\fill} }ccccc}
  \toprule
\textbf{Experiment} & \textbf{autotroph} & \textbf{herbivore} & \textbf{omnivore} & \textbf{carnivore} \\ 
  \midrule
1 & 3497787.37 & 242261.43 & 235485.41 & 213827.12 \\ 
  2 & 3497787.37 & 271421.40 & 264829.44 & 204436.57 \\ 
  3 & 3497787.37 & 271969.08 & 268020.89 & 78013.86 \\ 
  4 & 3497787.37 & 268435.81 & 267075.10 & 201853.16 \\ 
  5 & 3497787.37 & 310664.77 & 302392.36 & 44902.21 \\ 
  6 & 3497787.37 & 310836.82 & 306179.09 & 186371.46 \\ 
  7 & 3497787.37 & 310909.18 & 303400.83 & 44105.21 \\ 
  8 & 3497787.37 & 359780 & 356951.20 &  \\ 
  \cmidrule(lr){2-5}
   1 & 2399780.39 & 242127.95 & 236596.79 & 213827.12 \\ 
  2 & 2399780.39 & 271476.80 & 264984.12 & 203833.68 \\ 
  3 & 2399780.39 & 271969.08 & 268020.89 & 78013.86 \\ 
  4 & 2399780.39 & 268429.88 & 267075.10 & 202602.22 \\ 
  5 & 2399780.39 & 310057.76 & 302549.55 & 44883.61 \\ 
  6 & 2399780.39 & 311216.47 & 306451.11 & 187361.26 \\ 
  7 & 2399780.39 & 309274.79 & 302839.65 & 44211.79 \\ 
  8 & 2399780.39 & 360325.51 & 357285.42 &  \\ 
   \bottomrule
\end{tabular*}
\end{table}



\small
% latex table generated in R 3.1.3 by xtable 1.7-4 package
% Mon Jul 20 14:56:49 2015
\begin{table}[ht]
\centering
\begin{tabular*}{\textwidth}{@{\extracolsep{\fill} }ccccc}
  \toprule
& \multicolumn{2}{c}{\textbf{Cell 0}} & \multicolumn{2}{c}{\textbf{Cell 1}} \\
& \multicolumn{2}{c}{aseasonal} & \multicolumn{2}{c}{seasonal} \\
\cmidrule(lr){2-3} \cmidrule(lr){4-5}
\textbf{Experiments} & \textbf{Obs.} & \textbf{Crit.} & \textbf{Obs.} & \textbf{Crit.} \\ 
  \midrule
1-2 & 102.00 & 113.00 & 8.00 & 113.00 \\ 
  1-3 & 107.00 & 113.00 & 18.00 & 113.00 \\ 
  1-4 & \(\mathbf{422.00}\) & \(\mathbf{113.00}\) & 41.00 & 113.00 \\ 
  1-5 & 48.00 & 113.00 & 23.00 & 113.00 \\ 
  1-6 & \(\mathbf{538.00}\) & \(\mathbf{113.00}\) & 40.00 & 113.00 \\ 
  1-7 & \(\mathbf{490.00}\) & \(\mathbf{113.00}\) & 72.00 & 113.00 \\ 
  1-8 & \(\mathbf{554.00}\) & \(\mathbf{113.00}\) & 76.00 & 113.00 \\ 
   [1ex]2-3 & \(\mathbf{209.00}\) & \(\mathbf{113.00}\) & 26.00 & 113.00 \\ 
  2-4 & \(\mathbf{524.00}\) & \(\mathbf{113.00}\) & 49.00 & 113.00 \\ 
  2-5 & \(\mathbf{150.00}\) & \(\mathbf{113.00}\) & 31.00 & 113.00 \\ 
  2-6 & \(\mathbf{641.00}\) & \(\mathbf{113.00}\) & 48.00 & 113.00 \\ 
  2-7 & \(\mathbf{593.00}\) & \(\mathbf{113.00}\) & 80.00 & 113.00 \\ 
  2-8 & \(\mathbf{656.00}\) & \(\mathbf{113.00}\) & 84.00 & 113.00 \\ 
   [1ex]3-4 & \(\mathbf{315.00}\) & \(\mathbf{113.00}\) & 23.00 & 113.00 \\ 
  3-5 & 59.00 & 113.00 & 5.00 & 113.00 \\ 
  3-6 & \(\mathbf{431.00}\) & \(\mathbf{113.00}\) & 22.00 & 113.00 \\ 
  3-7 & \(\mathbf{383.00}\) & \(\mathbf{113.00}\) & 54.00 & 113.00 \\ 
  3-8 & \(\mathbf{447.00}\) & \(\mathbf{113.00}\) & 58.00 & 113.00 \\ 
   [1ex]4-5 & \(\mathbf{374.00}\) & \(\mathbf{113.00}\) & 18.00 & 113.00 \\ 
  4-6 & \(\mathbf{117.00}\) & \(\mathbf{113.00}\) & 1.00 & 113.00 \\ 
  4-7 & 69.00 & 113.00 & 31.00 & 113.00 \\ 
  4-8 & \(\mathbf{133.00}\) & \(\mathbf{113.00}\) & 35.00 & 113.00 \\ 
   [1ex]5-6 & \(\mathbf{490.00}\) & \(\mathbf{113.00}\) & 16.00 & 113.00 \\ 
  5-7 & \(\mathbf{442.00}\) & \(\mathbf{113.00}\) & 49.00 & 113.00 \\ 
  5-8 & \(\mathbf{506.00}\) & \(\mathbf{113.00}\) & 52.00 & 113.00 \\ 
   [1ex]6-7 & 48.00 & 113.00 & 32.00 & 113.00 \\ 
  6-8 & 16.00 & 113.00 & 36.00 & 113.00 \\ 
   [1ex]7-8 & 64.00 & 113.00 & 4.00 & 113.00 \\ 
   \bottomrule
\end{tabular*}
\caption[Kruskal-Wallis multiple comparison of autotroph biomass density.]{Results of \textit{post-hoc} Kruskal-Wallis multiple comparison
                tests between experiments for autotroph biomass density $[kgkm^{-2}]$ in the seasonal and aseasonal system. 
                  Data for each group consists of median values for the last 10 simulated years ($n_{i} = 121; \quad i = 1,\ldots8$). 
                  Significant differences between group medians, i.e. where observed 
                values are larger than the critical threshold ($\alpha = 0.055$) are given in bold.} 
\label{tab:chap:res:dyn:herbIND}
\end{table}

%
% latex table generated in R 3.1.3 by xtable 1.7-4 package
% Sat Aug 01 00:00:59 2015
\begin{table}[ht!]
\centering
\caption[Comparison of biomass density between cells]{Results of Mann-Whitney-U Tests comparing biomass density [$kg\cdot km^{-2}$] across experiments and 
               between cells ($n_i = 122, \quad with~i = 1,..8$) for all trophic groups (TG). Highly 
               significant results ($p \ll 0.05$) are marked with asterisks.} 
\label{tab:chap:res:comp}
\begin{tabular*}{\textwidth}{@{\extracolsep{\fill} }ccccccc}
  \toprule
\textbf{TG} & \textbf{Exp.} & \textbf{Cell 0} & \textbf{Cell 1} & \textbf{$U$} & \textbf{$P$} & \textbf{Sig.} \\ 
  \midrule
autotroph & 1 & 1670722.76 & 878695.31 & 12459 & $9.02\cdot 10^{-20}$ & *** \\ 
  autotroph & 2 & 1746510.83 & 882677.96 & 13740 & $3.18\cdot 10^{-30}$ & *** \\ 
  autotroph & 3 & 1561577.09 & 868540.39 & 11211 & $8.13\cdot 10^{-12}$ & *** \\ 
  autotroph & 4 & 486052.47 & 849876.63 & 4880 & $3.37\cdot 10^{-06}$ & *** \\ 
  autotroph & 5 & 1625253.14 & 868834.94 & 11774 & $3.92\cdot 10^{-15}$ & *** \\ 
  autotroph & 6 & 459224.36 & 853715.84 & 4880 & $3.37\cdot 10^{-06}$ & *** \\ 
  autotroph & 7 & 460777.43 & 832166.52 & 4880 & $3.37\cdot 10^{-06}$ & *** \\ 
  autotroph & 8 & 452182.73 & 835777.81 & 4880 & $3.37\cdot 10^{-06}$ & *** \\ 
   [1ex]carnivore & 1 & 73956.35 & 11956.17 & 14884 & $1.58\cdot 10^{-41}$ & *** \\ 
  carnivore & 2 & 128149.54 & 12114.82 & 14884 & $1.58\cdot 10^{-41}$ & *** \\ 
  carnivore & 3 & 84836.50 & 11900.03 & 14884 & $1.58\cdot 10^{-41}$ & *** \\ 
  carnivore & 4 & 16606.28 & 2235.03 & 14884 & $1.58\cdot 10^{-41}$ & *** \\ 
  carnivore & 5 & 145710.04 & 10875.68 & 14884 & $1.58\cdot 10^{-41}$ & *** \\ 
  carnivore & 6 & 1382.42 & 2185.08 & 5700 & 0.00158 & *** \\ 
  carnivore & 7 & 11067.04 & 0 & 14884 & $3.29\cdot 10^{-47}$ & *** \\ 
   [1ex]herbivore & 1 & 91839.98 & 289102.85 & 0 & $1.58\cdot 10^{-41}$ & *** \\ 
  herbivore & 2 & 66384.59 & 285987.31 & 0 & $1.58\cdot 10^{-41}$ & *** \\ 
  herbivore & 3 & 86550.09 & 287564.02 & 0 & $1.58\cdot 10^{-41}$ & *** \\ 
  herbivore & 4 & 371673.55 & 282915.75 & 14884 & $1.58\cdot 10^{-41}$ & *** \\ 
  herbivore & 5 & 69551.98 & 296088.09 & 0 & $1.58\cdot 10^{-41}$ & *** \\ 
  herbivore & 6 & 307365.73 & 281886.74 & 12575 & $1.27\cdot 10^{-20}$ & *** \\ 
  herbivore & 7 & 365987.03 & 285076.05 & 14884 & $1.58\cdot 10^{-41}$ & *** \\ 
  herbivore & 8 & 303425.90 & 291666.67 & 10402 & $7.93\cdot 10^{-08}$ & *** \\ 
   [1ex]omnivore & 1 & 14928.85 & 97738.93 & 0 & $1.58\cdot 10^{-41}$ & *** \\ 
  omnivore & 2 & 27758.63 & 99499.29 & 0 & $1.58\cdot 10^{-41}$ & *** \\ 
  omnivore & 3 & 21507.91 & 101026.99 & 0 & $1.58\cdot 10^{-41}$ & *** \\ 
  omnivore & 4 & 120955.38 & 110765.08 & 14712 & $1.04\cdot 10^{-39}$ & *** \\ 
  omnivore & 5 & 36271.91 & 100848.85 & 0 & $1.58\cdot 10^{-41}$ & *** \\ 
  omnivore & 6 & 185755.37 & 112323.49 & 14884 & $1.58\cdot 10^{-41}$ & *** \\ 
  omnivore & 7 & 131215.78 & 120835 & 14176 & $2.59\cdot 10^{-34}$ & *** \\ 
  omnivore & 8 & 187976.43 & 120333.03 & 14884 & $1.58\cdot 10^{-41}$ & *** \\ 
   \bottomrule
\end{tabular*}
\end{table}

%




\end{appendices}
