\section*{Abstract}
\label{Abstract}
\addcontentsline{toc}{chapter}{Abstract}
\onehalfspacing
Carnivore community composition and richness, i.e. which and how many functional groups are present, control trophic dynamics  and thereby regulate ecosystem functioning (ESF). Mechanisms underlying this regulation were mainly inferred via small-scale (temporal, spatial) experiments with restricted species pools. This limits their application to real-world systems.
This study tested if (i) these diversity effects are still found when experiments are vastly extended in scale and complexity to mimic natural systems, and (ii) whether climate altered these effects. 

Using the Madingley Model, two ecosystems (aseasonal, seasonal) were simulated over 100 years in eight experiments. These had either none ($n = 1$), a single ($ n = 3$), a combination of two ($n = 3$), or all (control, $n = 1$) of three carnivore functional groups present. These were defined by their metabolic and reproductive strategy - (1) ectothermic, iteroparous ($Ect_i$) or (2) semelparous ($Ect_s$), and (3) endothermic, iteroparous ($End_i$) – as well as maximum body mass ($Ect_i$ $>$ $End_i$ $\gg$ $Ect_s$).

$Ect_i$ presence determined autotroph biomass density, as a proxy for ESF, in the aseasonal system. Here, advantages given by lower relative metabolic costs (compared to $End_i$), constantly high ambient temperatures, and a large potential body mass range suppressed herbivory akin to control levels. \\
ESF remained virtually constant for the seasonal system, while community (abundance) and individual level (average body mass) properties shifted markedly across trophic groups and experiments. This implied that climatic productivity constraints determined ESF, while carnivore group composition and richness altered ecosystem structure. 


The results indicate that previously inferred mechanisms may only apply to subsets of natural systems. Research must be extended to include a variety of environments and generate more comprehensive understanding of biodiversity-ESF relationships, if management of ecosystems is to be successful in the face of climate change and increasing rates of biodiversity loss.






