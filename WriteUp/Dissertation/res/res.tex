\chapter{Results}
\label{chap:res}


\section{Trends vs. Steady State}
\label{chap:res:dyn:trend}

\begin{figure}
\centering
\includestandalone[width=\textwidth]{res/fig/ts_expno1}
\caption[Body mass density time series for experiment 1]{expno1 ts}
\label{fig:chap:res:ts:expno1}
\end{figure}

\begin{figure}
\centering
\includestandalone[width=\textwidth]{res/fig/BMD_ts_initial}
\caption[ShortCap]{}
\label{fig:chap:res:tsinit}
\end{figure}


\section{Ecosystem Dynamics}
\label{chap:res:dyn} 
differences between cell0 and cell1

\begin{figure}
\centering
\includestandalone[width=\textwidth]{res/fig/BMD_IND_0_avg}
\caption[Average body mass (aseasonal system)]{a}
\label{fig:chap:res:dyn:avg}
\end{figure}


\begin{figure}
\centering
\includestandalone[width=\textwidth]{res/fig/BMD_IND_0}
\caption{a}
\label{fig:chap:res:dyn}
\end{figure}


some text some text



%\subsection{Seasonality}
%\label{chap:res:dyn:seas}

this is for biomass density
%% latex table generated in R 3.1.3 by xtable 1.7-4 package
% Mon Jul 20 14:56:49 2015
\begin{table}[ht]
\centering
\begin{tabular*}{\textwidth}{@{\extracolsep{\fill} }ccccc}
  \toprule
& \multicolumn{2}{c}{\textbf{Cell 0}} & \multicolumn{2}{c}{\textbf{Cell 1}} \\
& \multicolumn{2}{c}{aseasonal} & \multicolumn{2}{c}{seasonal} \\
\cmidrule(lr){2-3} \cmidrule(lr){4-5}
\textbf{Experiments} & \textbf{Obs.} & \textbf{Crit.} & \textbf{Obs.} & \textbf{Crit.} \\ 
  \midrule
1-2 & 102.00 & 113.00 & 8.00 & 113.00 \\ 
  1-3 & 107.00 & 113.00 & 18.00 & 113.00 \\ 
  1-4 & \(\mathbf{422.00}\) & \(\mathbf{113.00}\) & 41.00 & 113.00 \\ 
  1-5 & 48.00 & 113.00 & 23.00 & 113.00 \\ 
  1-6 & \(\mathbf{538.00}\) & \(\mathbf{113.00}\) & 40.00 & 113.00 \\ 
  1-7 & \(\mathbf{490.00}\) & \(\mathbf{113.00}\) & 72.00 & 113.00 \\ 
  1-8 & \(\mathbf{554.00}\) & \(\mathbf{113.00}\) & 76.00 & 113.00 \\ 
   [1ex]2-3 & \(\mathbf{209.00}\) & \(\mathbf{113.00}\) & 26.00 & 113.00 \\ 
  2-4 & \(\mathbf{524.00}\) & \(\mathbf{113.00}\) & 49.00 & 113.00 \\ 
  2-5 & \(\mathbf{150.00}\) & \(\mathbf{113.00}\) & 31.00 & 113.00 \\ 
  2-6 & \(\mathbf{641.00}\) & \(\mathbf{113.00}\) & 48.00 & 113.00 \\ 
  2-7 & \(\mathbf{593.00}\) & \(\mathbf{113.00}\) & 80.00 & 113.00 \\ 
  2-8 & \(\mathbf{656.00}\) & \(\mathbf{113.00}\) & 84.00 & 113.00 \\ 
   [1ex]3-4 & \(\mathbf{315.00}\) & \(\mathbf{113.00}\) & 23.00 & 113.00 \\ 
  3-5 & 59.00 & 113.00 & 5.00 & 113.00 \\ 
  3-6 & \(\mathbf{431.00}\) & \(\mathbf{113.00}\) & 22.00 & 113.00 \\ 
  3-7 & \(\mathbf{383.00}\) & \(\mathbf{113.00}\) & 54.00 & 113.00 \\ 
  3-8 & \(\mathbf{447.00}\) & \(\mathbf{113.00}\) & 58.00 & 113.00 \\ 
   [1ex]4-5 & \(\mathbf{374.00}\) & \(\mathbf{113.00}\) & 18.00 & 113.00 \\ 
  4-6 & \(\mathbf{117.00}\) & \(\mathbf{113.00}\) & 1.00 & 113.00 \\ 
  4-7 & 69.00 & 113.00 & 31.00 & 113.00 \\ 
  4-8 & \(\mathbf{133.00}\) & \(\mathbf{113.00}\) & 35.00 & 113.00 \\ 
   [1ex]5-6 & \(\mathbf{490.00}\) & \(\mathbf{113.00}\) & 16.00 & 113.00 \\ 
  5-7 & \(\mathbf{442.00}\) & \(\mathbf{113.00}\) & 49.00 & 113.00 \\ 
  5-8 & \(\mathbf{506.00}\) & \(\mathbf{113.00}\) & 52.00 & 113.00 \\ 
   [1ex]6-7 & 48.00 & 113.00 & 32.00 & 113.00 \\ 
  6-8 & 16.00 & 113.00 & 36.00 & 113.00 \\ 
   [1ex]7-8 & 64.00 & 113.00 & 4.00 & 113.00 \\ 
   \bottomrule
\end{tabular*}
\caption[Kruskal-Wallis multiple comparison of autotroph biomass density.]{Results of \textit{post-hoc} Kruskal-Wallis multiple comparison
                tests between experiments for autotroph biomass density $[kgkm^{-2}]$ in the seasonal and aseasonal system. 
                  Data for each group consists of median values for the last 10 simulated years ($n_{i} = 121; \quad i = 1,\ldots8$). 
                  Significant differences between group medians, i.e. where observed 
                values are larger than the critical threshold ($\alpha = 0.055$) are given in bold.} 
\label{tab:chap:res:dyn:herbIND}
\end{table}



Median autotroph biomass ($[kg\cdot km^{-2}]$) and herbivore abundance ($[n\cdot km^{-2}$]) densities were compared between experiments (1 to 8) using a Krusikal-Wallis test with subsequent multiple comparison analysis (\add{adjusted p value?} $\alpha = 0.05$; cf. Table~\ref{tab:chap:res:dyn:autoBMD} and Table~\ref{tab:chap:res:dyn:herbIND}, respectively). 
Experiments (i.e. their mean ranks) differed significantly in biomass density response only for the aseasonal  system ($\chi^{2}(7) = 783.67$, $p \ll 0.05$). 
% seasonal  \add{adjusted p value?}($\chi^{2}(7) = 191.66$, $p \ll 0.05$) system.
\\
For abundance densities, significant differences between experiments were found for both the aseasonal ($\chi^{2}(7) = 834.34$, $p \ll 0.05$) and seasonal ($\chi^{2}(7) = 191.66$, $p \ll 0.05$) system. 
Note, the sampled experiments could not be considered having originated from identical distributions (cf.~Figure \add{HISTOGRAMS OF HERB. DENSITY ~ EXPNO in Appendix}); 
hence inferences base on mean ranks, rather than systematic shifts of central tendency (i.e. median) Figure~\add{Barplot BMD IDENS Densities}.

% latex table generated in R 3.1.3 by xtable 1.7-4 package
% Mon Jul 20 15:48:00 2015
\begin{table}[ht]
\centering
\small
\begin{tabular*}{\textwidth}{@{\extracolsep{\fill} }ccccc}
  \toprule
& \multicolumn{2}{c}{\textbf{Cell 0}} & \multicolumn{2}{c}{\textbf{Cell 1}} \\
& \multicolumn{2}{c}{aseasonal} & \multicolumn{2}{c}{seasonal} \\
\cmidrule(lr){2-3} \cmidrule(lr){4-5}
\textbf{Experiments} & \textbf{Obs.} & \textbf{Crit.} & \textbf{Obs.} & \textbf{Crit.} \\
  \midrule
  (1) & \multicolumn{2}{c}{(419.00)} & \multicolumn{2}{c}{(354.00)} \\
1-2 & \(\mathbf{195.00}\) & \(\mathbf{113.00}\) & 9.00 & 113.00 \\ 
  1-3 & \(\mathbf{258.00}\) & \(\mathbf{113.00}\) & \(\mathbf{148.00}\) & \(\mathbf{113.00}\) \\ 
  1-4 & \(\mathbf{305.00}\) & \(\mathbf{113.00}\) & 61.00 & 113.00 \\ 
  1-5 & 79.00 & 113.00 & \(\mathbf{114.00}\) & \(\mathbf{113.00}\) \\ 
  1-6 & \(\mathbf{388.00}\) & \(\mathbf{113.00}\) & 99.00 & 113.00 \\ 
  1-7 & \(\mathbf{130.00}\) & \(\mathbf{113.00}\) & \(\mathbf{341.00}\) & \(\mathbf{113.00}\) \\ 
  1-8 & \(\mathbf{459.00}\) & \(\mathbf{113.00}\) & \(\mathbf{325.00}\) & \(\mathbf{113.00}\) \\ 
   [1ex]
(2) & \multicolumn{2}{c}{(224.00)} & \multicolumn{2}{c}{(345.00)} \\   
   2-3 & \(\mathbf{454.00}\) & \(\mathbf{113.00}\) & \(\mathbf{157.00}\) & \(\mathbf{113.00}\) \\ 
  2-4 & 110.00 & 113.00 & 70.00 & 113.00 \\ 
  2-5 & \(\mathbf{274.00}\) & \(\mathbf{113.00}\) & \(\mathbf{122.00}\) & \(\mathbf{113.00}\) \\ 
  2-6 & \(\mathbf{583.00}\) & \(\mathbf{113.00}\) & 108.00 & 113.00 \\ 
  2-7 & 65.00 & 113.00 & \(\mathbf{350.00}\) & \(\mathbf{113.00}\) \\ 
  2-8 & \(\mathbf{654.00}\) & \(\mathbf{113.00}\) & \(\mathbf{333.00}\) & \(\mathbf{113.00}\) \\ 
   [1ex]
(3) & \multicolumn{2}{c}{(678.00)} & \multicolumn{2}{c}{(502.00)} \\   
   3-4 & \(\mathbf{563.00}\) & \(\mathbf{113.00}\) & 87.00 & 113.00 \\ 
  3-5 & \(\mathbf{180.00}\) & \(\mathbf{113.00}\) & 35.00 & 113.00 \\ 
  3-6 & \(\mathbf{130.00}\) & \(\mathbf{113.00}\) & 49.00 & 113.00 \\ 
  3-7 & \(\mathbf{388.00}\) & \(\mathbf{113.00}\) & \(\mathbf{193.00}\) & \(\mathbf{113.00}\) \\ 
  3-8 & \(\mathbf{200.00}\) & \(\mathbf{113.00}\) & \(\mathbf{176.00}\) & \(\mathbf{113.00}\) \\ 
   [1ex]
(4) & \multicolumn{2}{c}{(114.00)} & \multicolumn{2}{c}{(415.00)} \\   
   4-5 & \(\mathbf{384.00}\) & \(\mathbf{113.00}\) & 52.00 & 113.00 \\ 
  4-6 & \(\mathbf{693.00}\) & \(\mathbf{113.00}\) & 37.00 & 113.00 \\ 
  4-7 & \(\mathbf{175.00}\) & \(\mathbf{113.00}\) & \(\mathbf{280.00}\) & \(\mathbf{113.00}\) \\ 
  4-8 & \(\mathbf{764.00}\) & \(\mathbf{113.00}\) & \(\mathbf{263.00}\) & \(\mathbf{113.00}\) \\ 
   [1ex]
(5) & \multicolumn{2}{c}{(498.00)} & \multicolumn{2}{c}{(467.00)} \\   
   5-6 & \(\mathbf{309.00}\) & \(\mathbf{113.00}\) & 15.00 & 113.00 \\ 
  5-7 & \(\mathbf{208.00}\) & \(\mathbf{113.00}\) & \(\mathbf{228.00}\) & \(\mathbf{113.00}\) \\ 
  5-8 & \(\mathbf{380.00}\) & \(\mathbf{113.00}\) & \(\mathbf{211.00}\) & \(\mathbf{113.00}\) \\ 
   [1ex]
(6) & \multicolumn{2}{c}{(807.00)} & \multicolumn{2}{c}{(452.00)} \\    
   6-7 & \(\mathbf{518.00}\) & \(\mathbf{113.00}\) & \(\mathbf{243.00}\) & \(\mathbf{113.00}\) \\ 
  6-8 & 71.00 & 113.00 & \(\mathbf{226.00}\) & \(\mathbf{113.00}\) \\ 
   [1ex]
(7) & \multicolumn{2}{c}{(289.00)} & \multicolumn{2}{c}{(695.00)} \\   
   7-8 & \(\mathbf{588.00}\) & \(\mathbf{113.00}\) & 17.00 & 113.00 \\ 
   (8) & \multicolumn{2}{c}{(878.00)} & \multicolumn{2}{c}{(678.00)} \\
   \bottomrule
\end{tabular*}
\caption[Kruskal-Wallis multiple comparison of herbivore density.]{Results of \textit{post-hoc} Kruskal-Wallis multiple comparison
                            tests between experiments for herbivore density $[n\cdot km^{-2}]$ in the seasonal and aseasonal system.
                            Data for each group consists of the median values for the last 10 simulated years ($n_{i} = 121;\quad i = 1,\ldots8$).
                            Significant differences between groups, i.e. where observed (Obs.) aggregate differences exceeded the critical (Crit.) threshold ($\alpha = 0.055$), are given in bold. Values in parenthesis are median ranks.} 
\label{tab:chap:res:dyn:herbIND}
\end{table}



\begin{figure}
\centering
\includestandalone[width=\textwidth]{res/fig/BMD_IND_1_avg}
\caption[Average body mass (seasonal system)]{Cell1}
\label{fig:chap:res:dyn:cell1:avg}
\end{figure}


\begin{figure}
\centering
\includestandalone[width=\textwidth]{res/fig/BMD_IND_1}
\caption{Cell1}
\label{fig:chap:res:dyn:cell1}
\end{figure}


%%\begin{figure}
%%\centering
%%\includestandalone[width=\textwidth]{res/fig/comp_effects}
%%\caption{autotroph herbivore}
%%\label{fig:chap:res:dyn:compauto}
%%\end{figure}


\begin{figure}
\centering
\includestandalone[width=\textwidth]{res/fig/comp_effects_herb-herb}
\caption[Effect of carnivore group composition on herbivore biomass and abundance density]{herbivore herbivore}
\label{fig:chap:res:dyn:compherb}
\end{figure}


Test for experiment one ($Ect_i, Ect_s, End_i$) and experiment 7 ($End_i$) ($End_I, Ect_I$)

%
%% latex table generated in R 3.1.3 by xtable 1.7-4 package
% Mon Jul 20 00:22:14 2015
\begin{table}[ht]
\centering
\begin{tabular}{rll}
  \hline
 & Obs. & Crit. \\ 
  \hline
1-2 & 102 & 113 \\ 
  1-3 & 107 & 113 \\ 
  1-4 & \textbf{422} & \textbf{113} \\ 
  1-5 & 48 & 113 \\ 
  1-6 & \textbf{538} & \textbf{113} \\ 
  1-7 & \textbf{490} & \textbf{113} \\ 
  1-8 & \textbf{554} & \textbf{113} \\ 
  2-3 & \textbf{209} & \textbf{113} \\ 
  2-4 & \textbf{524} & \textbf{113} \\ 
  2-5 & \textbf{150} & \textbf{113} \\ 
  2-6 & \textbf{641} & \textbf{113} \\ 
  2-7 & \textbf{593} & \textbf{113} \\ 
  2-8 & \textbf{656} & \textbf{113} \\ 
  3-4 & \textbf{315} & \textbf{113} \\ 
  3-5 & 59 & 113 \\ 
  3-6 & \textbf{431} & \textbf{113} \\ 
  3-7 & \textbf{383} & \textbf{113} \\ 
  3-8 & \textbf{447} & \textbf{113} \\ 
  4-5 & \textbf{374} & \textbf{113} \\ 
  4-6 & \textbf{117} & \textbf{113} \\ 
  4-7 & 69 & 113 \\ 
  4-8 & \textbf{133} & \textbf{113} \\ 
  5-6 & \textbf{490} & \textbf{113} \\ 
  5-7 & \textbf{442} & \textbf{113} \\ 
  5-8 & \textbf{506} & \textbf{113} \\ 
  6-7 & 48 & 113 \\ 
  6-8 & 16 & 113 \\ 
  7-8 & 64 & 113 \\ 
   \hline
\end{tabular}
\end{table}
