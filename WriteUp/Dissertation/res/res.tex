\chapter{Results}
\label{chap:res}
Two ecosystems of distinct climatic character (i.e. equatorial, aseasonal and temperate, seasonal) were simulated over 100 years for different levels of carnivore functional group richness and composition (cf. Section~\ref{chap:mat:exp}). 
Resulting ecosystem functioning, using autotrophic biomass density as a proxy for biomass accumulation from primary production, is presented and related to community and individual level dynamics in higher trophic levels.
%
\section{Ecosystem Dynamics}
\label{chap:res:dyn} 
Discrepancies between both systems are apparent in distinct responses of autotroph biomass density for each experiment over time and once the respective systems have reached dynamic equilibrium.\\\\
Autotroph biomass density decreased significantly across levels of carnivore functional group richness in the aseasonal system (KW; $\chi^{2}_{(3)} = 345.71$ $p \ll 0.05$, cf. Table~\ref{tab:chap:res:dyn:autoBMD}) from approx. 1670723 $kg\cdot km^{-2}$ with all carnivore groups present, to approx. 452183 $kg\cdot km^{-2}$ without carnivores (cf. Table~\ref{tab:chap:res:median}). 
However, responses were not uniform (KW across experiments; $\chi^{2}_{(7)} = 783.67$ $p \ll 0.05$, cf. Table~\ref{tab:chap:res:dyn:autoBMD}), i.e. functional composition affected median biomass density within a given level of functional group richness. \\
In the seasonal system autotroph biomass density was unaffected by both carnivore functional group richness (KW; $\chi^{2}_{(3)} = 6.35$, $p = 0.096$) and composition (KW; $\chi^{2}_{(7)} = 10.08$, $p = 0.18$).

% latex table generated in R 3.1.3 by xtable 1.7-4 package
% Sat Aug 01 00:39:40 2015
\begin{table}[ht!]
\centering
\caption[Median biomass density at different levels of function group richness]{Median biomass density [$kg\cdot km^{-2}$] for both
               systems at different levels of carnivore functional group richness
               (FGR). LT and UT are the lower and upper tail of the 95~\%
               confidence interval from bootstrapping.} 
\label{tab:chap:res:median}
\begin{tabular*}{\textwidth}{@{\extracolsep{\fill} }ccccccc}
  \toprule
  & \multicolumn{3}{c}{Aseasonal} & \multicolumn{3}{c}{Seasonal} \\ \cmidrule{2-4} \cmidrule{5-7}
\textbf{FGR} & \textbf{Median} & \textbf{$-$LT} & \textbf{$+$UT} & \textbf{Median} & \textbf{$-$LT} & \textbf{$+$UT} \\ 
  \midrule
  9 & 1670723 & 78973 & 94978 & 878695 & 200314 & 267364 \\ 
    8 & 1561577 & 14399 & 15821 & 868540 & 202305 & 268966 \\ 
    7 & 460777 & 25127 & 21185 & 853716 & 203192 & 267288 \\ 
    6 & 452183 & 28740 & 15915 & 835778 & 227662 & 267998 \\ 
   \bottomrule
\end{tabular*}
\end{table}

\subsection{Development over time}
\label{chap:res:dyn:temporal}
%Overall dynamics are more strongly governed by climatic conditions in the temperate, seasonal ecosystem, whereas carnivore group composition dictates ecosystem functioning more pronouncedly for the aseasonal ecosystem 
Both systems showed temporal dynamics that can be related well to their climatic characteristics resulting from their respective latitiude (i.e. 0 for the aseasonal, and 52.5 for the seasonal system).\\
As expected, biomass densities in the temperate ecosystem follow seasonal cycles, evident in Figure~\ref{fig:chap:res:ts:expno1}.
 Annual turn-over reaches up to one order of magnitude for autotrophs, while less, but pronounced, variation is found for herbivores.
  Omnivores and carnivores closely follow these cycles  at lower biomass densities (respective order) and with smaller amplitudes (cf. Figure~\add{TS APPENDIX}). \\\\
  The aseasonal system is generally characterized by more stable biomass densities over time, while variation resulting from random organismal interactions encoded in the model, as opposed seasonal dynamics, dominates. 
  \textit{Exp. 4, 6, 7} and \textit{8} digress from this pattern, showing recurring cycles for autotroph and carnivore (where applicable) biomass densities, but far less  pronounced (lower amplitude) than found for the seasonal system. 
(cf. Figure~\ref{fig:chap:res:ts:expno1} and Figure~\ref{fig:chap:res:dyn:comp}
 \add{all TS APPENDIX}).\\\\
Both systems display drastic changes in median biomass density within their transient ("burn-in") phase within the first 15 to 25 and 10 years of the simulations, respectively, for all trophic groups (cf. Figure~\ref{fig:chap:res:ts:expno1}; exemplary). 
Subsequently, the systems develop towards a dynamic equilibrium. 
Autotroph and carnivore  biomass density in \textit{exp. 2} and \textit{exp. 5}, respectively, show slight trends over the last 50 years of the simulation. 
However, when applying the 90 year cut-off (used for subsequent analysis) these are negligible \add{appendix figure, all TS}. For \textit{exp. 7}, endothermic carnivores ($End_i$) show a rapid decline during the transient phase, leading to complete extinction in the seasonal system far before equilibrium was reached.
\begin{figure}[h!]
\centering
\includestandalone[width=\textwidth]{res/fig/ts_expno1}
\caption[Log-body mass density time series for \textit{exp. 1} in both systems]{Development of biomass density for the aseasonal (top) and seasonal (bottom) during 100 simulated years for autotrophs (green), herbivores (blue), omnivores (yellow) and carnivores (red). Lines are medians from 100 simulations and shaded areas represent the 95~\% confidence interval from bootstrapping. Both systems differ systematically regarding trophic level biomass dynamics: annual cycles are pronounced for all trophic groups in the seasonal system, with large (absolute) amplitudes found for autotrophs and herbivores. Variation for omnivores and carnivores is generally less pronounced (cf. \add{FIGURE TS APPEND}.) Fluctuations in the aseasonal system are more strongly governed by random organismal interaction, rather than being subject to climate forcing. Note, that the seemingly larger amplitude of carnivore biomass density (aseasonal system) is the result of using the logarithmic scale; absolute amplitudes are far larger for e.g. herbivores at a higher order of magnitude. }
\label{fig:chap:res:ts:expno1}
\end{figure}

\subsection{Climatic control and functional composition}
\label{chap:res:diff}
All focal trophic groups (autotrophs, herbivores, carnivores) had significantly different biomass densities between the simulated aseasonal and seasonal system for all variations of carnivore group composition or when all carnivores were removed from the system (Mann-Whitney-U Test at $\alpha = 0.05$, cf. Table~\ref{tab:chap:res:comp}).\\\\

Compared to the seasonal system, autotroph biomass density is considerably higher for \textit{exp. 1, 2, 3, and 5} in the aseasonal system, but lower for \textit{exp. 4, 6, 7 and 8} where ectothermic, iteroparous carnivores ($Ect_i$) were absent. As described in Section~\ref{chap:res:dyn}, autotroph biomass density remains at approx. constant values across experiments for the seasonal system (taking into account annual fluctuations). 
Analogous, herbivore biomass density is approx. constant, while the aseasonal system mirrors the pattern found for autotrophs (i.e. low when high, and vice-versa). \\\\
Except for \textit{exp. 6}, carnivore biomass density is considerably higher for the aseasonal system. 
However, the varying carnivore group composition leads to a distinct pattern, where low (or zero) carnivore biomass density for both systems coincide (e.g. \textit{exp 4, 6, 7, 8}), relating well to the patterns of autotroph and herbivore biomass density for the aseasonal system. \\
Yet, as described above, no similar directional trends are apparent for the seasonal system, and biomass densities for the lower trophic groups (i.e. autotrophs and herbivores), are not affected by the decrease in carnivore biomass density. \add{Discussion? This lack of downward cascading effects suggest that either (i) climate takes precedence  over carnivore group composition in controlling ecosystem functioning in the simulated temperate system or (ii) that there is a minimum threshold of carnivore biomass density required for exerting top-down control.}

\begin{figure}
\centering
\includestandalone[width=\textwidth]{res/fig/BMD_Cell1-2_AHC}
\caption[Comparison median biomass density (BMD) for aseasonal and seasonal ecosystems]{Median biomass densities (BMD) at equilibrium (final 10 years) between the seasonal and aseasonal systems for autotrophs (A), herbivores B) and carnivores (C) across all experiments (1 to 8). Error bars are 95~\% confidence intervals. Labels under bars signify carnivore group richness and composition. BMD of all depicted trophic groups differ significantly (marked with asterisks) between systems for all experiments ($p << 0.05$, Mann-Whitney-U Test at $\alpha = 0.05$, cf. Table~\ref{tab:chap:res:comp}).% Autotroph BMD is considerably higher for \textit{exp. 1, 2, 3, and 5} in the aseasonal system, but lower for \textit{exp. 4, 6, 7 and 8}, while remaining at approx. constant values throughout all experiments for the seasonal system. Analogous, herbivore BMD are constant across experiments, with the aseasonal system mirroring values for autotrophic biomass density (i.e. low when high, and vice-versa). Except for \textit{exp. 6}, carnivore BMD is considerably higher throughout, showing varied responses in the aseasonal system, while remaining fairly constant in the seasonal system. However, experiments lead to a consistent pattern, where low (or zero) carnivore BMD for both systems coincide (e.g. \textit{exp 4, 6, 7, 8})
 A consistent emergent pattern of low (or zero) carnivore BMD coinciding with high herbivore and low autotroph BMD in \textit{exp. 4, 6, 7} is evident for the aseasonal system.
%  While carnivore BMD responds similar across experiments in the seasonal system (i.e. decrease compared to \textit{exp. 1}),
   There is no apparent cascading effect from carnivores in the seasonal system  to lower trophic levels, as both herbivore and autotroph BMD remain approx. constant across experiments.
  This hints at seasonality overriding the role of carnivore composition (and consequently any top-down control) in determining ecosystem functioning. }
\label{fig:chap:res:dyn:comp}
\end{figure}




\section{Community and individual level response}
\label{chap:res:popind}
%
\textbf{trophic pyramid for seasonal, only for some in aseasonalb}
\add{Discussion? Community level (abundance density) and individual (average body mass) properties govern the response of a trophic group on ecosystem level (biomass density), and thereby ecosystem functioning. 
That is, the ecosystem level property considered here will only be affected when changes in average body mass coincide with an appropriate directional shift of overall abundance in the respective trophic group (e.g. heavier organisms at higher abundances mean higher body mass densities)}.
\subsection{Aseasonal system}
\label{chap:res:popind:cell0}
% all of the aforementioned experiments have the absence of ectothermic, iteroparous carnivores ($Ect_i$) in common. 
Herbivore biomass densities were significantly different between experiments  (KW; $\chi^{2}_{(7)} = 929.94$, $p \ll 0$), with \textit{exp. 4, 6 and 7} as well as carnivore absence (\textit{exp. 8)} leading to higher herbivore biomass density. 
This, in turn, cascaded further downward decreasing autotroph biomass density (cf. Figure~\ref{fig:chap:res:dyn:cell0}A).
%Experiments (i.e. their mean ranks) differed significantly in biomass density response only for the aseasonal  system ($\chi^{2}(7) = 783.67$, $p \ll 0.05$). 
%% seasonal  \add{adjusted p value?}($\chi^{2}(7) = 191.66$, $p \ll 0.05$) system.
%\\
\\\\
This pattern, i.e. high herbivore biomass densities for \textit{exp. 4, 6, 7} and \textit{8}, is the result of shifting community (abundance density) and individual (average body mass) properties in response to altered carnivore functional group composition (cf. Figures~\ref{fig:chap:res:dyn:cell0}B and \ref{fig:chap:res:dyn:cell0}C): when ectothermic, iteroparous carnivores ($Ect_i$) are present, i.e. \textit{exp. 1, 2, 3} and \textit{5}, average herbivore body mass remains below approx. 0.2~$kg\cdot n^{-1}$ \add{ADD TABLES OF AVBM, IDENS}. 
Coinciding with intermediate abundances (cf. Figure~\ref{fig:chap:res:dyn:cell0}B), 
this leads to herbivore biomass densities that maintain the overall ecosystem functioning at levels akin to \textit{exp. 1} (control with all carnivore groups present).\\\\
Herbivore average body mass increases for the remaining experiments to at least 0.3~$kg\cdot n^{-1}$ (\textit{exp. 6}, $Ect_s$) and reaches a maximum of approx. 1.2~$kg\cdot n^{-1}$ for \textit{exp. 4}  ($Ect_s$ and $End_i$). 
When endothermic carnivores ($End_i$) are present (\textit{exp. 4} and \textit{7}, this increase is not accompanied by higher abundances, but still results in higher biomass density (i.e. same abundance, but heavier organisms). 
In contrast, \textit{exp. 6} ($Ect_s$) leads to significantly higher abundances of relatively low average body mass individuals and therefore to increased herbivore biomass density, only matched by \textit{exp. 8} (no carnivores present).


For   system. 
%Note, the sampled experiments could not be considered having originated from identical distributions (cf.~Figure \add{HISTOGRAMS OF HERB. DENSITY ~ EXPNO in Appendix}); 
%hence inferences base on mean ranks, rather than systematic shifts of central tendency (i.e. median) Figure~\add{Barplot BMD IDENS Densities}.

\begin{figure}
\centering
\includestandalone[width=\textwidth]{res/fig/BMD_IND_AVBM_0}
\caption[Ecosystem, community and individual level response to altered carnivore group richness and composition in the aseasonal system]{Overview of ecosystem, community and individual level herbivore dynamics in response to altered carnivore group richness and composition in the aseasonal system. Error bars are 95~\% confidence intervals from boot-strapping. Labels under bars signify carnivore group richness and composition. (A) Biomass density (KW; $\chi^{2}_{(7)} = 929.94$, $p \ll 0$), (B) abundance densities (KW; $\chi^{2}_{(7)} = 834.34$, $p \ll 0.05$) and (C) average body mass (KW; $\chi^{2}_{(7)} = 875.89$, $p \ll 0.05$) were significantly different across experiments. Herbivore biomass density is governed by average body mass of individuals in concert with abundance density (community level property). Absence of ectothermic, iteroparous carnivores ($Ect_i$; \textit{exp. 4, 6, 7} and \textit{8}) markedly alters herbivore average body mass and abundance, resulting in lower ecosystem functioning (autotroph biomass density). Letters above bars signify significant differences between experiments from \textit{post-hoc} multiple comparisons. }
\label{fig:chap:res:dyn:cell0}
\end{figure}



% latex table generated in R 3.1.3 by xtable 1.7-4 package
% Mon Jul 20 14:56:49 2015
\begin{table}[ht]
\centering
\begin{tabular*}{\textwidth}{@{\extracolsep{\fill} }ccccc}
  \toprule
& \multicolumn{2}{c}{\textbf{Cell 0}} & \multicolumn{2}{c}{\textbf{Cell 1}} \\
& \multicolumn{2}{c}{aseasonal} & \multicolumn{2}{c}{seasonal} \\
\cmidrule(lr){2-3} \cmidrule(lr){4-5}
\textbf{Experiments} & \textbf{Obs.} & \textbf{Crit.} & \textbf{Obs.} & \textbf{Crit.} \\ 
  \midrule
1-2 & 102.00 & 113.00 & 8.00 & 113.00 \\ 
  1-3 & 107.00 & 113.00 & 18.00 & 113.00 \\ 
  1-4 & \(\mathbf{422.00}\) & \(\mathbf{113.00}\) & 41.00 & 113.00 \\ 
  1-5 & 48.00 & 113.00 & 23.00 & 113.00 \\ 
  1-6 & \(\mathbf{538.00}\) & \(\mathbf{113.00}\) & 40.00 & 113.00 \\ 
  1-7 & \(\mathbf{490.00}\) & \(\mathbf{113.00}\) & 72.00 & 113.00 \\ 
  1-8 & \(\mathbf{554.00}\) & \(\mathbf{113.00}\) & 76.00 & 113.00 \\ 
   [1ex]2-3 & \(\mathbf{209.00}\) & \(\mathbf{113.00}\) & 26.00 & 113.00 \\ 
  2-4 & \(\mathbf{524.00}\) & \(\mathbf{113.00}\) & 49.00 & 113.00 \\ 
  2-5 & \(\mathbf{150.00}\) & \(\mathbf{113.00}\) & 31.00 & 113.00 \\ 
  2-6 & \(\mathbf{641.00}\) & \(\mathbf{113.00}\) & 48.00 & 113.00 \\ 
  2-7 & \(\mathbf{593.00}\) & \(\mathbf{113.00}\) & 80.00 & 113.00 \\ 
  2-8 & \(\mathbf{656.00}\) & \(\mathbf{113.00}\) & 84.00 & 113.00 \\ 
   [1ex]3-4 & \(\mathbf{315.00}\) & \(\mathbf{113.00}\) & 23.00 & 113.00 \\ 
  3-5 & 59.00 & 113.00 & 5.00 & 113.00 \\ 
  3-6 & \(\mathbf{431.00}\) & \(\mathbf{113.00}\) & 22.00 & 113.00 \\ 
  3-7 & \(\mathbf{383.00}\) & \(\mathbf{113.00}\) & 54.00 & 113.00 \\ 
  3-8 & \(\mathbf{447.00}\) & \(\mathbf{113.00}\) & 58.00 & 113.00 \\ 
   [1ex]4-5 & \(\mathbf{374.00}\) & \(\mathbf{113.00}\) & 18.00 & 113.00 \\ 
  4-6 & \(\mathbf{117.00}\) & \(\mathbf{113.00}\) & 1.00 & 113.00 \\ 
  4-7 & 69.00 & 113.00 & 31.00 & 113.00 \\ 
  4-8 & \(\mathbf{133.00}\) & \(\mathbf{113.00}\) & 35.00 & 113.00 \\ 
   [1ex]5-6 & \(\mathbf{490.00}\) & \(\mathbf{113.00}\) & 16.00 & 113.00 \\ 
  5-7 & \(\mathbf{442.00}\) & \(\mathbf{113.00}\) & 49.00 & 113.00 \\ 
  5-8 & \(\mathbf{506.00}\) & \(\mathbf{113.00}\) & 52.00 & 113.00 \\ 
   [1ex]6-7 & 48.00 & 113.00 & 32.00 & 113.00 \\ 
  6-8 & 16.00 & 113.00 & 36.00 & 113.00 \\ 
   [1ex]7-8 & 64.00 & 113.00 & 4.00 & 113.00 \\ 
   \bottomrule
\end{tabular*}
\caption[Kruskal-Wallis multiple comparison of autotroph biomass density.]{Results of \textit{post-hoc} Kruskal-Wallis multiple comparison
                tests between experiments for autotroph biomass density $[kgkm^{-2}]$ in the seasonal and aseasonal system. 
                  Data for each group consists of median values for the last 10 simulated years ($n_{i} = 121; \quad i = 1,\ldots8$). 
                  Significant differences between group medians, i.e. where observed 
                values are larger than the critical threshold ($\alpha = 0.055$) are given in bold.} 
\label{tab:chap:res:dyn:herbIND}
\end{table}

%% latex table generated in R 3.1.3 by xtable 1.7-4 package
% Mon Jul 20 15:48:00 2015
\begin{table}[ht]
\centering
\small
\begin{tabular*}{\textwidth}{@{\extracolsep{\fill} }ccccc}
  \toprule
& \multicolumn{2}{c}{\textbf{Cell 0}} & \multicolumn{2}{c}{\textbf{Cell 1}} \\
& \multicolumn{2}{c}{aseasonal} & \multicolumn{2}{c}{seasonal} \\
\cmidrule(lr){2-3} \cmidrule(lr){4-5}
\textbf{Experiments} & \textbf{Obs.} & \textbf{Crit.} & \textbf{Obs.} & \textbf{Crit.} \\
  \midrule
  (1) & \multicolumn{2}{c}{(419.00)} & \multicolumn{2}{c}{(354.00)} \\
1-2 & \(\mathbf{195.00}\) & \(\mathbf{113.00}\) & 9.00 & 113.00 \\ 
  1-3 & \(\mathbf{258.00}\) & \(\mathbf{113.00}\) & \(\mathbf{148.00}\) & \(\mathbf{113.00}\) \\ 
  1-4 & \(\mathbf{305.00}\) & \(\mathbf{113.00}\) & 61.00 & 113.00 \\ 
  1-5 & 79.00 & 113.00 & \(\mathbf{114.00}\) & \(\mathbf{113.00}\) \\ 
  1-6 & \(\mathbf{388.00}\) & \(\mathbf{113.00}\) & 99.00 & 113.00 \\ 
  1-7 & \(\mathbf{130.00}\) & \(\mathbf{113.00}\) & \(\mathbf{341.00}\) & \(\mathbf{113.00}\) \\ 
  1-8 & \(\mathbf{459.00}\) & \(\mathbf{113.00}\) & \(\mathbf{325.00}\) & \(\mathbf{113.00}\) \\ 
   [1ex]
(2) & \multicolumn{2}{c}{(224.00)} & \multicolumn{2}{c}{(345.00)} \\   
   2-3 & \(\mathbf{454.00}\) & \(\mathbf{113.00}\) & \(\mathbf{157.00}\) & \(\mathbf{113.00}\) \\ 
  2-4 & 110.00 & 113.00 & 70.00 & 113.00 \\ 
  2-5 & \(\mathbf{274.00}\) & \(\mathbf{113.00}\) & \(\mathbf{122.00}\) & \(\mathbf{113.00}\) \\ 
  2-6 & \(\mathbf{583.00}\) & \(\mathbf{113.00}\) & 108.00 & 113.00 \\ 
  2-7 & 65.00 & 113.00 & \(\mathbf{350.00}\) & \(\mathbf{113.00}\) \\ 
  2-8 & \(\mathbf{654.00}\) & \(\mathbf{113.00}\) & \(\mathbf{333.00}\) & \(\mathbf{113.00}\) \\ 
   [1ex]
(3) & \multicolumn{2}{c}{(678.00)} & \multicolumn{2}{c}{(502.00)} \\   
   3-4 & \(\mathbf{563.00}\) & \(\mathbf{113.00}\) & 87.00 & 113.00 \\ 
  3-5 & \(\mathbf{180.00}\) & \(\mathbf{113.00}\) & 35.00 & 113.00 \\ 
  3-6 & \(\mathbf{130.00}\) & \(\mathbf{113.00}\) & 49.00 & 113.00 \\ 
  3-7 & \(\mathbf{388.00}\) & \(\mathbf{113.00}\) & \(\mathbf{193.00}\) & \(\mathbf{113.00}\) \\ 
  3-8 & \(\mathbf{200.00}\) & \(\mathbf{113.00}\) & \(\mathbf{176.00}\) & \(\mathbf{113.00}\) \\ 
   [1ex]
(4) & \multicolumn{2}{c}{(114.00)} & \multicolumn{2}{c}{(415.00)} \\   
   4-5 & \(\mathbf{384.00}\) & \(\mathbf{113.00}\) & 52.00 & 113.00 \\ 
  4-6 & \(\mathbf{693.00}\) & \(\mathbf{113.00}\) & 37.00 & 113.00 \\ 
  4-7 & \(\mathbf{175.00}\) & \(\mathbf{113.00}\) & \(\mathbf{280.00}\) & \(\mathbf{113.00}\) \\ 
  4-8 & \(\mathbf{764.00}\) & \(\mathbf{113.00}\) & \(\mathbf{263.00}\) & \(\mathbf{113.00}\) \\ 
   [1ex]
(5) & \multicolumn{2}{c}{(498.00)} & \multicolumn{2}{c}{(467.00)} \\   
   5-6 & \(\mathbf{309.00}\) & \(\mathbf{113.00}\) & 15.00 & 113.00 \\ 
  5-7 & \(\mathbf{208.00}\) & \(\mathbf{113.00}\) & \(\mathbf{228.00}\) & \(\mathbf{113.00}\) \\ 
  5-8 & \(\mathbf{380.00}\) & \(\mathbf{113.00}\) & \(\mathbf{211.00}\) & \(\mathbf{113.00}\) \\ 
   [1ex]
(6) & \multicolumn{2}{c}{(807.00)} & \multicolumn{2}{c}{(452.00)} \\    
   6-7 & \(\mathbf{518.00}\) & \(\mathbf{113.00}\) & \(\mathbf{243.00}\) & \(\mathbf{113.00}\) \\ 
  6-8 & 71.00 & 113.00 & \(\mathbf{226.00}\) & \(\mathbf{113.00}\) \\ 
   [1ex]
(7) & \multicolumn{2}{c}{(289.00)} & \multicolumn{2}{c}{(695.00)} \\   
   7-8 & \(\mathbf{588.00}\) & \(\mathbf{113.00}\) & 17.00 & 113.00 \\ 
   (8) & \multicolumn{2}{c}{(878.00)} & \multicolumn{2}{c}{(678.00)} \\
   \bottomrule
\end{tabular*}
\caption[Kruskal-Wallis multiple comparison of herbivore density.]{Results of \textit{post-hoc} Kruskal-Wallis multiple comparison
                            tests between experiments for herbivore density $[n\cdot km^{-2}]$ in the seasonal and aseasonal system.
                            Data for each group consists of the median values for the last 10 simulated years ($n_{i} = 121;\quad i = 1,\ldots8$).
                            Significant differences between groups, i.e. where observed (Obs.) aggregate differences exceeded the critical (Crit.) threshold ($\alpha = 0.055$), are given in bold. Values in parenthesis are median ranks.} 
\label{tab:chap:res:dyn:herbIND}
\end{table}



\subsection{Seasonal System}
\label{chap:res:popind:cell1}
 For the seasonal system no pronounced dynamics were found regarding changes of autotroph or herbivore biomass density (cf. Figure~\ref{fig:chap:res:dyn:cell1}), even though there were significant differences for the latter (KW; $\chi^{2}_{(7)} = 48$, $p \ll 0.05$). It must be noted that this result should be taken with caution, as tested groups were inhomogeneous regarding their distributions; detected differences may be over-estimated, which is hinted at by highly similar median values and confidence intervals in Figure~\ref{fig:chap:res:dyn:cell1}A.
\\\\
While the overall ecosystem functioning (i.e. biomass density) in the seasonal system was not strongly affected, at community level abundance density was significantly higher at lower levels of carnivore functional group richness (KW; $\chi^{2}_{(3)} = 115.84$, $p \ll 0.05$ ) and differed  with carnivore group composition (KW; $\chi^{2}_{(7)} = 191.66$, $p \ll 0.05$).
\\ 
Consequently (as biomass densities remained fairly constant), individuals had significantly lower average body mass at lower levels of carnivore functional group richness (KW; $\chi^{2}_{(3)} = 129.86$, $p \ll 0.05$); carnivore group composition lead to significantly different average body masses within these levels (KW; $\chi^{2}_{(7)} = 221$ $p \ll 0.05$. \\
As no carnivores were present when \textit{exp. 7} reached dynamic equilibrium, this experiment must be considered mechanistically identical to to \textit{exp. 8} (no carnivores present). Results  regarding functional group richness - and inferences thereby - should be therefore be taken with caution.\\\\
For the seasonal system, the presence of ectothermic, semelparous carnivores, in any combination, is associated with a size structure (i.e. herbivore $>$ omnivore $>$ carnivore) similar to \textit{exp. 1} (control). Their absence therefore coincides with higher herbivore abundance density (cf. Figures~\ref{fig:chap:res:dyn:cell1}B and \ref{fig:chap:res:dyn:cell1}C).
 
 

\begin{figure}
\centering
\includestandalone[width=\textwidth]{res/fig/BMD_IND_AVBM_1}
\caption[Ecosystem, community and individual level response to altered carnivore group richness and composition in the seasonal system]{Overview of ecosystem, community and individual level herbivore dynamics in response to altered carnivore group richness and composition in the seasonal system. Error bars are 95~\% confidence intervals from boot-strapping. Labels under bars signify carnivore group richness and composition. (A) Biomass density (KW; $\chi^{2}_{(7)} = 48$ $p \ll 0.05$), (B) abundance density ($\chi^{2}(7) = 191.66$, $p \ll 0.05$) and (C) average body mass (KW; $\chi^{2}_{(7)} = 221$, $p \ll 0.05$). Biomass density remains fairly - if not entirely (cf. text) - constant across experiments, while abundance density and average body mass increase and decrease, respectively, with functional group richness. The absence of ectothermic, iteroparous carnivores ($Ect_s$) results in higher herbivore abundances and an altered size structure in \textit{exp. 3} and \textit{6}. Total carnivore absence (\textit{exp. 7, 8}) leads to high herbivore abundances at lowest average body mass. Letters above bars signify significant differences between experiments from \textit{post-hoc} multiple comparisons.}
\label{fig:chap:res:dyn:cell1}
\end{figure}


%%\begin{figure}
%%\centering
%%\includestandalone[width=\textwidth]{res/fig/comp_effects}
%%\caption{autotroph herbivore}
%%\label{fig:chap:res:dyn:compauto}
%%\end{figure}


%\begin{figure}
%\centering
%\includestandalone[width=\textwidth]{res/fig/comp_effects_herb-herb}
%\caption[Effect of carnivore group composition on herbivore biomass and abundance density]{herbivore herbivore}
%\label{fig:chap:res:dyn:compherb}
%\end{figure}
\section{Comparison to empirical observations}
\add{to Discussion:
Empirical estimates of heterotroph-autotroph ratios for the terrestrial realm are generally sparse. 
Unsurprisingly, the locations for the obtained values do not match the simulated ecosystems.}\\\\
Empirical estimates of heterotroph - autotroph ratios were taken from a compilation in the supplemental material in \cite{Harfoot2014}. Generally, all predicted estimates from the model are far greater than empirical values for ecosystems of comparable latitude (cf. Figure~\ref{fig:chap:res:dyn:ratios}). \\Herbivore - autotroph ratios in the control (\textit{exp. 1}) and experiments with similar outcomes (\textit{exp. 2, 3} and \textit{5}) in the aseasonal system have the highest resemble with the empirical estimate found for an ecosystem with the closest match in latitude (i.e. Serengeti, latitude $  \approx -2.33$), while the remainder are one to two orders of magnitude greater. \\\\
The seasonal system shows herbivore - autotroph ratios that are approx. constant across all experiments, and lie several orders of magnitude above both higher and lower latitude  systems found in North America (approx. 20$^\circ$ difference). \\ 
The only available carnivore - autotroph ratios are at least three orders of magnitude below the model prediction.
\begin{figure}
\centering
\includestandalone[width=\textwidth]{res/fig/HetAut-Ratios}
\caption[Comparison to empirical data]{comparison}
\label{fig:chap:res:dyn:ratios}
\end{figure}



%% latex table generated in R 3.1.3 by xtable 1.7-4 package
% Mon Jul 20 00:22:14 2015
\begin{table}[ht]
\centering
\begin{tabular}{rll}
  \hline
 & Obs. & Crit. \\ 
  \hline
1-2 & 102 & 113 \\ 
  1-3 & 107 & 113 \\ 
  1-4 & \textbf{422} & \textbf{113} \\ 
  1-5 & 48 & 113 \\ 
  1-6 & \textbf{538} & \textbf{113} \\ 
  1-7 & \textbf{490} & \textbf{113} \\ 
  1-8 & \textbf{554} & \textbf{113} \\ 
  2-3 & \textbf{209} & \textbf{113} \\ 
  2-4 & \textbf{524} & \textbf{113} \\ 
  2-5 & \textbf{150} & \textbf{113} \\ 
  2-6 & \textbf{641} & \textbf{113} \\ 
  2-7 & \textbf{593} & \textbf{113} \\ 
  2-8 & \textbf{656} & \textbf{113} \\ 
  3-4 & \textbf{315} & \textbf{113} \\ 
  3-5 & 59 & 113 \\ 
  3-6 & \textbf{431} & \textbf{113} \\ 
  3-7 & \textbf{383} & \textbf{113} \\ 
  3-8 & \textbf{447} & \textbf{113} \\ 
  4-5 & \textbf{374} & \textbf{113} \\ 
  4-6 & \textbf{117} & \textbf{113} \\ 
  4-7 & 69 & 113 \\ 
  4-8 & \textbf{133} & \textbf{113} \\ 
  5-6 & \textbf{490} & \textbf{113} \\ 
  5-7 & \textbf{442} & \textbf{113} \\ 
  5-8 & \textbf{506} & \textbf{113} \\ 
  6-7 & 48 & 113 \\ 
  6-8 & 16 & 113 \\ 
  7-8 & 64 & 113 \\ 
   \hline
\end{tabular}
\end{table}
