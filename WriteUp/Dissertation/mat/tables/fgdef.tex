% latex table generated in R 3.1.3 by xtable 1.7-4 package
% Sun Aug 02 11:46:40 2015
\begin{sidewaystable}[htb!]
\centering
\caption[Functional group definitions]{Functional group ($n = 9$) definitions based on categorical traits (trophic type, reproductive and metabolic strategy),
               and additional parameters that determine the ecology of a group (proportional assimilation efficiency and the proportion of each time step
               at which an organism is active). Maximum and minimum body mass determine the size range for the cohorts seeded into the model at the initial time step ($n = 112$ per functional group). Dietery specialisation is considered to result in higher assimilation efficiencies, i.e. obligate herbivores and carnivores gain more energy from a suitable food source than do omnivores. For this study, carnivore group richness and composition was altered (cf. Section~\ref{chap:mat:exp}).} 
\label{tab:mat:madingley:func}
\begin{tabulary}{\textwidth}{CCCCCCCR}
  \toprule
  & & & \multicolumn{2}{c}{\thead{Prop. Assimilation \\ Efficiency}} & & \multicolumn{2}{c}{\thead{$\log_{10}$ Body Mass \\ Range [$kg$]}} \\ \cmidrule(lr){4-5} \cmidrule(lr){7-8} 
\textbf{Trophic Type} & \textbf{Reproduction} & \textbf{Metabolism} & \textbf{Herbivore} & \textbf{Carnivore} & \textbf{Activity} & \textbf{Min.} & \textbf{Max.} \\ 
  \midrule
Carnivore & itero. & Endotherm & 0.00 & 0.80 & 0.50 & -2.52 & 2.85 \\ 
  Carnivore & semel. & Ectotherm & 0.00 & 0.80 & 0.50 & -6.10 & 0.30 \\ 
  Carnivore & itero. & Ectotherm & 0.00 & 0.80 & 0.50 & -2.82 & 3.30 \\ 
   [1ex]Herbivore & itero. & Endotherm & 0.50 & 0.00 & 0.50 & -2.82 & 3.70 \\ 
  Herbivore & semel. & Ectotherm & 0.50 & 0.00 & 0.50 & -6.40 & 0.00 \\ 
  Herbivore & itero. & Ectotherm & 0.50 & 0.00 & 0.50 & -3.00 & 2.48 \\ 
   [1ex]Omnivore & itero. & Endotherm & 0.40 & 0.64 & 0.50 & -2.52 & 3.18 \\ 
  Omnivore & semel. & Ectotherm & 0.40 & 0.64 & 0.50 & -6.40 & 0.30 \\ 
  Omnivore & itero. & Ectotherm & 0.40 & 0.64 & 0.50 & -2.82 & 1.74 \\ 
   \bottomrule
\end{tabulary}
\end{sidewaystable}
