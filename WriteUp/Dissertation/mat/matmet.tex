\chapter{Materials and Methods}
\label{chap:mat}
The following sections give a brief overview to the    structure and functioning of the \textsc{Madingley Model}, as well as the experimental design (i.e. simulated experiments), data preparation and subsequent analyses used in this study. Model details and explanations are taken from the main text and supplemental material of \cite{Harfoot2014} unless other sources are specified, and generally refer to processes in terrestrial ecosystems only.\\\\
 All data and computer code to reproduce the results presented in the Section~\ref{chap:res} can be found in the attached data medium and at \textit{www.github.com/the-Hull} (as of July, 2015). The model, along with supplemental material detailing technical specifications and underlying equations are also found on the storage.
\section{The Madingley Model}
\label{chap:mat:madingley}
%
The \textsc{Madingley Model} is the first general ecosystem model of its kind. 
It is process-based  and mechanistic. (i.e. uses fundamental ecological principles; cf. Section~\ref{chap:mat:madingley:princip}) and achieves its generality by being applicable to both the terrestrial and marine realm, while including both autotroph and heterotroph dynamics across all trophic levels. \\
Interactions as well as individual  dynamics for organisms of all sizes (i.e. spanning several orders of magnitude) can be simulated at high temporal and spatial resolutions. 
Ecosystem dynamics are determined by local environmental conditions and interactions with surrounding systems (i.e. grid cells). 
Rather than relying on a species specific definition of organisms, the model uses a trait-based approach, where an organism's role is determined by its membership within a functional group (cf. Section~\ref{chap:mat:madingley:func}).
\\\\
Environmental conditions (for terrestrial ecosystems: precipitation
\citetalias{FC2012}, air temperature \citetalias{FC2012}, frost days \citetalias{FC2012}, available soil water \citetalias{ISRIC2012}  and seasonality of NPP \citetalias{NASA2012}) are sourced for the grid cells corresponding to the simulated systems  from open-access databases. As of now, the model does not simulate freshwater ecosystems within the terrestrial realm.
\subsection{Functional groups, stocks and cohort dynamics}
\label{chap:mat:madingley:func}
%
Organisms are characterized by fixed functional traits, which determine functional group membership. These are simple, categorical life history traits, such as feeding mode or reproductive strategy (cf. Table~\ref{tab:mat:madingley:func}). In addition, there are continuous traits, such as body mass or assimilation efficiencies for food intake, which modulate the ecological processes encoded into the model (cf. Section~\ref{chap:mat:madingley:princip}), along with a user-defined body mass range (adopted from \citealp{Harfoot2014}).
% latex table generated in R 3.1.3 by xtable 1.7-4 package
% Sun Aug 02 11:46:40 2015
\begin{sidewaystable}[htb!]
\centering
\caption[Functional group definitions]{Functional group ($n = 9$) definitions based on categorical traits (trophic type, reproductive and metabolic strategy),
               and additional parameters that determine the ecology of a group (proportional assimilation efficiency and the proportion of each time step
               at which an organism is active). Maximum and minimum body mass determine the size range for the cohorts seeded into the model at the initial time step ($n = 112$ per functional group). Dietery specialisation is considered to result in higher assimilation efficiencies, i.e. obligate herbivores and carnivores gain more energy from a suitable food source than do omnivores. For this study, carnivore group richness and composition was altered (cf. Section~\ref{chap:mat:exp}).} 
\label{tab:mat:madingley:func}
\begin{tabulary}{\textwidth}{CCCCCCCR}
  \toprule
  & & & \multicolumn{2}{c}{\thead{Prop. Assimilation \\ Efficiency}} & & \multicolumn{2}{c}{\thead{$\log_{10}$ Body Mass \\ Range [$kg$]}} \\ \cmidrule(lr){4-5} \cmidrule(lr){7-8} 
\textbf{Trophic Type} & \textbf{Reproduction} & \textbf{Metabolism} & \textbf{Herbivore} & \textbf{Carnivore} & \textbf{Activity} & \textbf{Min.} & \textbf{Max.} \\ 
  \midrule
Carnivore & itero. & Endotherm & 0.00 & 0.80 & 0.50 & -2.52 & 2.85 \\ 
  Carnivore & semel. & Ectotherm & 0.00 & 0.80 & 0.50 & -6.10 & 0.30 \\ 
  Carnivore & itero. & Ectotherm & 0.00 & 0.80 & 0.50 & -2.82 & 3.30 \\ 
   [1ex]Herbivore & itero. & Endotherm & 0.50 & 0.00 & 0.50 & -2.82 & 3.70 \\ 
  Herbivore & semel. & Ectotherm & 0.50 & 0.00 & 0.50 & -6.40 & 0.00 \\ 
  Herbivore & itero. & Ectotherm & 0.50 & 0.00 & 0.50 & -3.00 & 2.48 \\ 
   [1ex]Omnivore & itero. & Endotherm & 0.40 & 0.64 & 0.50 & -2.52 & 3.18 \\ 
  Omnivore & semel. & Ectotherm & 0.40 & 0.64 & 0.50 & -6.40 & 0.30 \\ 
  Omnivore & itero. & Ectotherm & 0.40 & 0.64 & 0.50 & -2.82 & 1.74 \\ 
   \bottomrule
\end{tabulary}
\end{sidewaystable}

In total there are nine functional groups implemented in the model, three for each of the heterotrophic groups - herbivore, omnivore and carnivore. Plants are represented as, stocks with which herbivores and omnivores can interact.\\\\
 Individuals are grouped into cohorts, which act unanimously (i.e. as individuals).  A cohort is characterized by fixed traits, i.e. functional group membership, body mass at birth and at reproductive maturity, as well as three traits that are updated within each time step, and are affected by the ecological processes (cf. Section~\ref{chap:mat:madingley:princip}): individual abundance within a cohort, current (wet matter) body mass, and stored reproductive body mass. \\
 The cohort approach was adopted to decrease computational costs. In this light, merging was also implemented, where cohorts of similar characteristics are combined, when a certain threshold number of cohorts is reached. This ensures computational tractability and has been shown to have little to no effect on outcomes of modelled ecosystems \citep[cf.][]{Harfoot2014}.
 \clearpage
\subsection{Encoded Ecological Principles}
\label{chap:mat:madingley:princip}
The developers encoded into the model a set of ecological and biological processes they deemed appropriate for simulating ecosystem structure and functioning. These include: primary production (autotrophs), metabolism, eating, growth, reproduction, dispersal and mortality (heterotrophs). For detailed equations, see supplemental material (S1 \textsc{Model Equations}). All processes occur in random order each time step, and determine the total abundance and biomass within a cohort and grid cell in the next time step. \\\\
\textbf{Primary production} (i.e. plant growth) is based on an explicit carbon stock model using remotely sensed estimates of net primary productivity developed by \cite{Smith2012}. 
Plants allocate a fixed proportion of their growth into structural maintenance, while the remainder is invested in other tissue, such as leaves (deciduous, evergreen) and fruits, which are edible for herbivores and omnivores. 
\\\\
\textbf{Heterotrophic growth} is determined by the difference of energy assimilated from food via herbivory (herbivores, omnivores) or predation (omnivores, carnivores) and metabolic loss, as well as energy allocated for reproduction.\\\\
\textbf{Eating} is defined as herbivory for herbivores and omnivores, and as predation for omnivores and carnivores, where both plant stocks and heterotroph cohorts are well mixed within a grid cell. All organisms spend a fixed amount of time foraging; for omnivores this is equally distributed between herbivory and predation. 
The efficiency with which food is assimilated is a fundamental parameter defined for each functional group (cf. Section~\ref{chap:mat:madingley:func} and Table~\ref{tab:mat:madingley:func}). \\
Herbivory and predation are implemented via a Holling's Type III response function, which assumes that food intake (plant or autotroph tissue) by an organism depends on prey (or plant) density \citep{Denno2012}. 
This is further constrained by foraging processes, i.e. handling time and attack rates, which are allometrically scaled for herbivory, and are based on a size-structured model for predation by omnivores and carnivores \citep[after][; cf. supplemental material for details]{Williams2011}. 
That is, predation only occurs on cohorts of suitable size classes proportional to predator body mass.
\\
 Handling time for predation found in omnivores and carnivores increases linearly with prey body mass, but decreases as predators increase in body size (power law relationship). Herbivore handling times are only determined by their respective body mass. \\\\
\textbf{Metabolism} is modelled according to  power-law relationships which scale with body mass after \cite{Brown2004}, based on field and basal metabolic rates derived from \cite{Nagy1999}. 
 Activity is governed by ambient temperature, and hence thermoregulating organisms (endotherms) are assumed to be active throughout each simulated time step. 
\\\\
\textbf{Reproduction} occurs after an individual has reached adult body mass, when all further body mass is stored for reproduction. 
Once enough reproductive potential has been allocated (i.e. body mass), an organism will bear offspring. 
Semelparous organisms use all of their stored reproductive biomass and part of their adult biomass, while iteroparous organisms only make use of their stored reproductive biomass. Offspring is generated in new cohorts.
\\\\
\textbf{Mortality} is governed by four processes: (i) a background mortality, implemented as a constant rate; (ii) starvation, when an organisms loses a certain proportion of its maximum body mass;  (iii) senescence, which increases exponentially once an organism has reached maturity (i.e. adult body mass); and (iv) predation. \\\\
\textbf{Dispersal} is always experienced by the entire cohort (i.e. a cohort is never split). For this study, two mechanisms of dispersal (within grid cells) were modelled: (i) active, but random, diffusive dispersal after birth for juvenile cohorts in search of new territory; and (ii) active dispersal when a certain threshold of  food deprivation (i.e. low resource densities) has been experienced. As only single grid cells were simulated, there was no dispersal in or out of the regarded system.
 
%\subsection{Model structure, input and definitions}
%\label{chap:mat:madingley:structure}
%
%\subsection{Outputs}
%\label{chap:mat:madingley:output}

\section{Experiment Design}
\label{chap:mat:exp}
Two ecosystems with distinct climatic conditions, one highly productive, aseasonal in Southern Uganda and one temperate, seasonal system in Central England (cf. Table~\ref{tab:mat:exp:loc}) were simulated in individual $1^\circ$ by $1^\circ$ cells (i.e. no dispersal in or out of the system was permitted). 
Climate data and environmental conditions were taken from the sources mentioned in Section~\ref{chap:mat:madingley}, and implemented as long-term monnthly averages and repeated for each year.\\
% latex table generated in R 3.1.3 by xtable 1.7-4 package
% Sun Aug 02 13:02:07 2015
\begin{table}[htb!]
\centering
\caption[Specific locations and characteristics of the modelled ecosystems]{Specific locations and characteristics of the modelled ecosystems.} 
\label{tab:mat:exp:loc}
\begin{tabular*}{\textwidth}{@{\extracolsep{\fill} } ccccc}
  \toprule
\textbf{Location} & \textbf{Latitude} & \textbf{Longitude} & \textbf{Cell Code} & \textbf{Character} \\ 
  \midrule
South Ugunda & 0.00 & 32.50 & Cell 0 & aseasonal \\ 
  Central England & 52.50 & 0.50 & Cell 1 & seasonal \\ 
   \bottomrule
\end{tabular*}
\end{table}

These locations are in concert with those used in \citet{Harfoot2014} and were chosen to elucidate the effect of climate on ecosystem functioning and to allow comparisons with previous results. 
\\
Simulations were run for 100 years (monthly resolution) to allow for the development of a dynamic equilibrium; 
this was based on prior experience (oral communication with model developers) and previous results from \cite{Harfoot2014}.
\\\\
The role of carnivore functional group richness and composition was determined by simulating both systems under different levels of richness and composition: 
the first experiment was done with all three carnivore groups present ($Ect_i, Ect_s, End_i$) as a control (\textit{exp. 1}), 6 subsequent experiments had either combinations of two groups (\textit{exp. 2 -- 4}), or only one group present (\textit{exp. 5 -- 7}). 
Finally, one set of simulations with no carnivores present was used to identify if a certain experiment resulted in dynamics akin to a system devoid of any carnivores (\textit{exp. 8}). Table~\ref{tab:mat:exp} gives an overview of the experimental design. \\
Experiments were repeated 100 times to account for variability resulting from stochastic elements in the model (i.e. dispersal, predation).
% latex table generated in R 3.1.3 by xtable 1.7-4 package
% Sun Aug 02 12:35:58 2015
\begin{table}[htb!]
\centering
\caption[Functional group definitions]{Experiments, corresponding carnivore functional group composition and richness. Both systems were simulated in 100 replicates over 100 years for each experiment. Ectohermic, iteroparous and semelparous organisms are labelled $Ect_i$ and $Ect_s$ respectively, and endothermic, iteroparous organisms as $End_i$. These labels are used for reference in the following sections.} 
\label{tab:mat:exp}
\begin{tabular*}{\textwidth}{@{\extracolsep{\fill} } cccc}
  \toprule
\textbf{Experiment} & \textbf{Composition} & \textbf{Richness} & \textbf{Initial Cohorts} \\ 
  \midrule
  1 & $End_i$ + $Ect_s$ + $Ect_i$ & 3 & 336 \\ 
   [1ex]  2 & $Ect_i$ + $Ect_s$ & 2 & 224 \\ 
    3 & $End_i$ + $Ect_i$ & 2 & 224 \\ 
    4 & $End_i$ + $Ect_s$ & 2 & 224 \\ 
   [1ex]  5 & $Ect_i$ & 1 & 112 \\ 
    6 & $Ect_s$ & 1 & 112 \\ 
    7 & $End_i$ & 1 & 112 \\ 
   [1ex]  8 & (-) & 0 & 0 \\ 
   \bottomrule
\end{tabular*}
\end{table}

\\\\
Ecosystem functioning was measured as biomass density ([$kg\cdot km^{-2}$]) of the basal trophic level (autotrophs), and related to ecosystem (biomass density), community (abundance density [$n\cdot km^{-2}$]) and individual level (average body mass [[$kg\cdot n^{-1}$]) properties of strict herbivores. This rationale is based on the assumption that altered carnivore functional diversity (i.e. richness or composition) would result in cascading effects from herbivores to autotrophs. These measures are also presented for omnivores and carnivores alike for completeness and interpretation. Results of statistical analysis of these are not included in the main body for conciseness, but can be found in the appendix.
%
\subsection{Model set-up and initialisation}
For the first time step, the user-specified body mass range is randomly sampled (uniform distribution) to produce parameters (e.g. body mass at birth) for the initial set of  cohorts (cf. Section~\ref{chap:mat:madingley:func}).
The simulated ecosystems were seeded with 112 of these randomly produced cohorts (user specified, cf. \citealp{Harfoot2014}) per grid cell for each functional group present in an experiment.\\
Initial cohort numbers were not adjusted to control levels (i.e. \textit{exp. 1}) across experiments with lower functional richness (cf. Figure~\ref{fig:app:tsinit}). This was done to emulate the effect of full extinction (i.e. instantaneous loss of entire carnivore groups) on an intact system.\\
All parameters used in the simulations are found in the supplemental material. 

%Terminology, Cohort seeding: Figure~\ref{fig:app:tsinit}
%\section{Model Set-up}
%

\section{Analysis}
\label{chap:mat:analysis}
All analyses were done using the statistical programming language \textsc{R} v3.1.3 \citep{RCT2015}. All other necessary software packages can be found in the supplemental material and are mentioned if crucial for the analyses.
\subsection{Data preparation and preliminary analyses}
The simulations were run for 100 years at monthly resolution, resulting in 1201 (including initial time step $t = 0$) observations per functional group of biomass and abundance density for each experiment. Median values for each time step were then calculated from replicates ($n = 100$). 
\\\\
For the statistical analyses (cf. Section~\ref{chap:mat:analysis:stats}) a representative sample of median biomass and abundance density was drawn from the final years of the simulation, assuming that the systems (for each experiment) had reached dynamic equilibrium with respect to ecosystem functioning (i.e. biomass density). 
Observations within these samples were considered independent, as dynamics at equilibrium are quasi-stable and remain within limits given by climatic and ecological constraints imposed by the experiments. \\
The time frame from which to draw the samples was determined using a simple procedure: (1) discard the first half of the data from the transient ("burn-in") phase; (2) fit simple linear regression models (biomass density $\sim$ time) for a specified window (e.g. from year 51 to 100) for all functional groups  across all experiments and record the number of significant slopes (i.e. trends); (3) repeat step (2) incrementally decreasing the scanning window (e.g. 52 to 100 years) until a minimum window size of 5 years (ensuring adequate sample size) is reached; (4) choose the year with the lowest number of significant trends as the first year of the representative sample.\\
Year 90 was  identified as the optimal lower end of the sampling window; applying this universally ensured equal sample sizes for each experiment ($n = 122$).\\\\
Average body mass was calculated by dividing  biomass density by abundance density for each time step in the equilibrium sample.

\subsection{Statistical Analyses}
\label{chap:mat:analysis:stats}
Samples were highly heterogeneous regarding their distributions (e.g. variance, skewness) within a trophic group when comparing experiments and systems. None of the tested transformations lead to compliance with assumptions for parametric tests. Hence, all analyses were done using either Mann-Whitney-U Tests \citep{Hollander2013} for assessing differences between systems for a given experiment and trophic group, or 
Kruskal-Wallis H Tests with \textit{post-hoc} multiple comparison tests (R package \textsc{pgirmess}, v1.6.2; \citealp{Siegel1988}) for determining differences between experiments for a trophic group (i.e for functional richness and composition). \\\\
For each presented median value 95~\% confidence intervals were estimated using a simple percentile boot-strapping approach as presented in \cite{Carpenter2000}: the observations used for calculating the median were re-sampled (with replacement) 10000 times, and boot-strap medians calculated for each. The 2.75 and 97.5~\% percentiles of the boot-strapped median set were then set as the upper and lower limits of confidence interval, respectively. 
\section{Comparison with Empirical Data}
\label{chap:mat:emp}
Model predictions of heterotroph - autotroph biomass ratios were compared with values derived from \cite{Cebrian2009}, which are found in the supplemental material of \cite{Harfoot2014}.
% The locations for the empirical estimates are given in Table~	 \add{Empirical locations table}

