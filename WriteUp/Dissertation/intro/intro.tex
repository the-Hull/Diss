\chapter{Introduction}
\label{chap:intro}
\textbf{State of environment / biodiversity loss}\\
Undoubtedly, one of the most unsettling truths of our time is that we cannot accurately predict how our actions will affect ecosystems and their functioning on regional, let alone planetary scale.\\
Accelerating global change resulting from various human pressures, such as overexploitation, climate change and habitat destruction, is threatening the persistence of ecosystems and consequently human well-being \citep{MEA2005, Hooper2012}. In fact, these pressures, have already lead to a worrisome loss of biodiversity (\add{perhaps glossary?})
with extinction rates rapidly increasing for various taxonomic groups \citep{Dirzo2003,Wake2008}.
%in 2014 the IUCN (International Union for Conservation of Nature) listed 22413 of its 76199 assessed species as threatened by extintion \citep{IUCN2014}. 
Predictions of biodiversity loss are equally grim: in a meta-analysis including results from 131 publications, \cite{Urban2015} estimated that climate change - as a singular driver - would threaten 16~\% of the planet's species with extinction under a business-as-usual scenario (average temperature increase of 4.3~$^{\circ}C$). 
%This is especially disturbing as the current expert-consensus on reaching the international policy goal of limiting temperature increase to a 2~$^{\circ}C$ compared to pre-industrial conditions can no longer be achieved \citep{Fuss2014}.  
\\\\
%
\textbf{Importance of BD for EF}\\
 This \emph{extinction wave}, termed by \cite{Barnosky2011} as a potential sixth mass extinction event, is likely to impair ecosystem functioning and ecosystem service provisioning \citep{MEA2005,Naeem2012}, as
%Biodiversity is a major determinant of ecosystem functioning, expressed through processes such as primary production, nutrient cycling, energy transfer between and within trophic levels, as well as responses to changing environmental conditions; 
%recent research consistently suggests that 
biodiversity loss decreases the rates at which essential processes (e.g. primary production, nutrient cycling, energy transfer between trophic levels) occur \citep[][]{Cardinale2012, Hooper2012}. \\
Ecosystem functioning - or corresponding processes - have been shown to saturate with increasing biodiversity, potentially leading to a steep decrease or even cessation of an ecosystem process under continued extinction \citep{Cardinale2012,Tilman2014}. 
Even though this pattern (concave-down) has generally been observed for simplified, experimental and agricultural systems, the concave-up relationship between biodiversity and ecosystem functioning found in natural systems could render effects of biodiversity loss even more detrimental \citep{Cardinale2012,Mora2014}. \\\\
%
\textbf{Keystone Species / BEF}\\
Negative impacts may be further exacerbated, as functionally important species (\emph{keystone species}) are often more prone to extinction due to inherent traits, such as slow reproductive cycles and low population densities \citep{Cardillo2005}, selective pressures, such as hunting by humans \citep{Dobson2006}, or simply due to low abundances, restricted distributions or susceptibility to stochastic events \citep{MacArthur1967,Smith2003}. For example,
% ungulates in the African savanna maintain the ecosystem's structure by controlling primary production, yet their replacement with functionally less diverse cattle has caused widespread degradation of these ecosystems \citep{DuToit1999}; 
apex predators, such as wolves or lions, are often subject to selective hunting or extirpation \citep{Dobson2006,Ripple2014}, yet they can be of disproportionate importance for ecosystem functioning, when the top-down regulation they impose on a system's trophic structure is stronger than opposing bottom-up effects \citep{Borer2006,Estes2011}.
 \\\\
 %
 \textbf{Hisory of BEF Research}\\
Effects of biodiversity loss on biodiversity-ecosystem functioning (BEF) relationships were mainly inferred experimentally or via theoretical models by investigating diversity effects in random plant assemblages \citep[e.g.][]{Tilman2001,Hector2007,Isbell2011}, or between adjacent trophic levels \citep[mostly plant-herbivore interactions, e.g.][]{Thebault2003,Bruno2008} with limited species pools, commonly using species richness as a biodiversity metric. Theory underlying this research attributed biodiversity effects on ecosystem functioning to two mechanisms: 
(1) complementarity (i.e. niche partitioning and facilitation), as the spectrum of available resources is more fully exploited while more species with differing strategies or requirements establish,
 and (2) selection effects, where single or few highly productive species dominate communities due to the prevalence of certain traits \citep{Loreau2001,Tilman2001}.\\
Biodiversity loss was generally emulated by decreasing the number of randomly chosen species per assemblage or simulation \citep{Tilman2001,Thebault2003,Ives2005}, while trophic structuring often served as a proxy of ecosystem functioning by measuring total biomass production and energy transfer between groups of organisms \citep[cf.][]{Tilman2014}.
  \add{species removal experiments, observational studies of BD loss}. \\\\
%
Some early research focused on ecosystem functioning in more complex systems, such observational study by \cite{Estes1974} on overgrazing of kelp forests by sea urchins after sea otter abundance had been greatly reduced, or work by \cite{Sih1998} elucidating how multi-species interactions shape ecosystem functioning. 
Several more recent studies report multi-trophic effects cascading from predators to plants for select ecosystems, where increasing predator diversity indirectly enhanced primary producer performance in salt marshes \citep{Finke2004}, kelp forests \citep{Byrnes2006} and a number  of terrestrial ecosystems in western North America \citep{Beschta2009} by (experimentally) releasing herbivore pressure. 
\add{multi-species predator-prey interactions \citep[cf.][]{Ives2005}}. 

%
However, species may be functionally redundant regarding an ecosystem process \citep{Hooper2002}, which can potentially mask diversity effects on ecosystem functioning when using species richness as the biodiversity metric \citep{Mora2014}. This realization, along with other difficulties that can arise in large scale experimental or observational studies in natural systems \citep[e.g. misidentification or low encounter probabilities during sampling;][]{Martinez1999}, lead to the emergence and application of functional diversity for experimental, observational and simulation-based studies. Functional diversity (FD) is a measure of biodiversity based on traits (e.g. photosynthetic pathway, foraging strategy) and/or trophic identity, akin to the niche and guild concept \citep{Hooper2002}. \add{Examples of Functional Diversity Research and findings, add \citep{Wu2015} for recent multi-trophic study}
%
%Indeed, over the last decades, considerable effort has gone into BEF research investigating single and multi-trophic systems using  functional diversity, a metric based on traits (e.g. photosynthetic pathway, foraging strategy) and trophic identity, akin to the niche and guild concept \citep{Hooper2002}. 
\\\\
%
Species richness is not easily determined in real-world systems, as establishing exhaustive inventories is nigh impossible - species can be misidentified or may have a low encounter probability during sampling; yet rare species can have an important role for ecosystem functioning \citep{Jain2013}. Further, species are often functionally redundant, theoretically making a trait-based approach for measuring biodiversity-ecosystem functioning relationships more feasible  \\\\
%More recently, attention has shifted towards identifying effects of horizontal (within a trophic level) and vertical (length of trophic chain) diversity \citep{Duffy2007}  on ecosystem functioning, acknowledging that intra-guild predation and omnivory govern the strength and can alter the sign of observed diversity effects \citep{Ives2005}. 


%
\textbf{Non-Random extinction}\\
\rephrase{paragraph structure doesn't make sense}\\
As outlined above, humanly driven extinctions can be non-random.
Indeed, a number of empirical \citep{Sih1998,OConnor2008} and simulation-based\citep{Borrvall2000,Ives2005} studies considered the effect of (\textbf{include non-random loss here?}) loss of predators on trophic structuring or BEF with consistent results: 
decreased predator diversity results in higher abundances of herbivores (and potentially omnivores) increasing pressure on primary producers, while intraguild predation and omnivory dampen these cascading effects \citep{Duffy2007}. \add{non-random species citations: \cite{Fox2012,Wardle2011,Saguin2014}} \\\\
%Trophic structuring often served as a proxy of ecosystem functioning by measuring total biomass production and energy transfer between groups of organisms. 
%
\textbf{Shortcomings of experimental studies}\\
However, by design, these studies are restricted to simple experimental or observational settings, few interacting agents (i.e. species or functional groups) and are constrained in space and time \citep{Dobson2006,Estes2011,Cardinale2012}; this stands true even for comparatively long-lived experiments, such as the 20 year Cedar Creek experiment in Minnesota \citep[][]{Tilman2001} or large-scale experiments, such as the IGRE species removal experiment in Mongolia, China \citep{Wu2015}.\\
While these approaches guarantee tractability and often generate mechanistic understanding, their results can hardly be extrapolated to regional or global scales \citep{Bulling2006,Naeem2012}, and they lack the ability to capture transient effects of e.g highly mobile agents that connect ecosystems functionally \citep{France2006}. \\
\rephrase{structure!}
and their ability to address questions such as the impact of non-random species loss from selective pressures on BEF in real-world or large-scale ecosystems \citep{Saguin2014,Wardle2011},
Additionally, a comprehensive understanding of BEF relationships and how ecosystems are functionally linked would require extensive research effort, as most regions, especially those at highest risk of biodiversity loss, are understudied - often due to their remoteness (e.g. Amazonia), yet carry disproportionally high numbers of endemic species \citep{Urban2015,Wake2008}. While these studies 
\\\\
%Evidently, inferring BEF relationships from natural ecosystems (opposed to simple experimental or agricultural systems)  via experimental manipulation - and extrapolating to regional or global scales - may be impossible, as reaching adequate temporal and spatial scales to account for environmental variability is infeasible \citep{Duffy2007}. \\\\
%
%For example, an observational study by \cite{Estes1974} reported overgrazing of kelp forests by sea urchins after sea otter abundance had been greatly reduced. 
%Multi-trophic effects (cascades) from predators to plants were found in several ecosystem types, where increasing predator diversity indirectly enhanced primary producer performance in salt marshes \citep{Finke2004}, kelp forests \citep{Byrnes2006} and a number  of terrestrial ecosystems in western North America \citep{Beschta2009} by (experimentally) releasing herbivore pressure. \add{multi-species predator-prey interactions \citep{Ives2005}}.\\\\
%
\textbf{Use of Models}\\
Predicting the impact of biodiversity loss urgently requires mechanistic understanding of biodiversity-ecosystem functioning (BEF) relationships applicable to 'real-world' systems across the planet \citep{Dobson2006,Estes2011}. \add{\cite{Naeem2012}: identifies need for models}
%

\section{Biodiversity and Ecosystem functioning}
\label{chap:intro:biodiv}
Here we go  here wo changes