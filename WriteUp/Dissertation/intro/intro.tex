\chapter{Introduction}
\label{chap:intro}
% Accelerating global change resulting from anthropogenic pressures, such as overexploitation, climate change and habitat destruction is driving grave losses of biodiversity \cite{Wake2008, Ceballos2015}. 
Biodiversity is a major determinant of ecosystem functioning, expressed through processes such as primary production, nutrient cycling, as well as energy transfer between and within trophic levels \citep[cf.][for review]{Tilman2014}. 
Current and predicted species losses due to accelerating global change \citep[e.g.][]{Ceballos2015, Urban2015} therefore threaten not only the persistence of ecosystems, but also human well-being due to our dependence on services derived from nature \citep{MEA2005,Cardinale2012}. .
\\\\
Recognizing the importance of biodiversity for maintaining ecosystem functioning has spurred great efforts in the field of biodiversity-ecosystem functioning (BEF) research in the past decades.  A comprehensive review by \cite{Cardinale2012} allows concluding that diversity enhances processes at all levels of biological and ecological organisation.  Dynamics within and across trophic levels, especially, have been shown to regulate ecosystem functioning to a large degree \citep{Duffy2007, Estes2011}.  \\\\
Hence, predators have received extensive attention. This is not only because they are generally faced with higher extinction risk compared to organisms on lower trophic levels  \citep{Dobson2006,Ripple2014}. But, more so, because they exert (on average) stronger top-down control on an ecosystem, than bottom-up control arises from changes in basal trophic levels \citep{Borer2006}.  \\\\
Recent studies have shown that, next to predator richness, the presence (or absence) of certain predators can have a decisive effect on ecosystem processes by modulating trophic cascades through intra-guild predation \citep{Finke2005,Wallach2015}, competition for resources  \citep{OConnor2008,Rodriguez-Lozano2015}, or combinations of both (amongst others). %These effects were clearly linked to traits that allow organisms to fulfil one (or more) roles in their respective ecosystem.
 %and thereby define functional group membership \citep{Hooper2002}. 
\\\\ 
%As a result, research in biodiversity-ecosystem functioning (and related areas) has been vigorously pursued in the last decades to unravel underlying mechanisms.\\
%To date, multiple meta-analyses have confirmed that ecosystem functioning increases with biodiversity, and that high levels of biodiversity can buffer functions against species losses \citep{Balvanera2006, Cardinale2006, Hooper2012}. 
However, these conclusions are generally drawn from controlled experimental environments and/or inferred from limited species pools, in single, or across (mainly adjacent) trophic levels \citep[e.g.][]{Finke2005, Byrnes2006}.  \\
Their basis in unnatural systems is a fundamental - though necessary - limitation regarding their transfer to and general applicability in real-world ecosystems \citep{Hillebrand2009}. 
%even though the importance of some identified effects may be gravely under-estimated \citep{Duffy2009}.
 \\\\ 
 Unsurprisingly, two developments in ecology have been called for repeatedly in the literature regarding future management of ecosystems \citep[e.g.][]{Hooper2012,Naeem2012,Tilman2014}: \\
 (1) gaps in mechanistic understanding of how biodiversity regulates ecosystem functioning in variable, heterogeneous landscapes through time must be filled; and (2) a universal tool or framework that allows predicting how ecosystems will respond to changing conditions, especially in the face of increasing anthropogenic pressures and biodiversity loss \citep{Wardle2011}, need to be developed.\\\\
 %; and 
% Unsurprisingly, the need of predicting how  ecosystems under varying environmental conditions will respond to anthropogenic pressures, and especially biodiversity loss \citep{Wardle2011}, has been stated repeatedly \citep[e.g.][]{Hooper2012,Naeem2012,Tilman2014}.\\
%  While new mechanistic understanding of BEF relationships is still necessarily being generated via experimental and modelling approaches to fill existing gaps \citep[e.g.][]{Li2014,Lefcheck2015}, until recently a widely applicable framework or tool for predicting ecosystem functioning was still missing \citep{Purves2013,Mace2013}.\\\\
\cite{Harfoot2014} addressed this need, and introduced a general, mechanistic ecosystem model (\textsc{Madingley Model}) that can simulate highly complex multi-trophic systems (i.e. individual interactions between and across trophic levels) in both marine and terrestrial environments while taking local environmental and climatic conditions into account. \\\\
The model relies on functional groups, which circumvents having to identify each agent taxonomically, and therefore even allows simulating unstudied ecosystems. 
Simple organismal and life history traits, such as body mass and trophic identity  \citep[][shown to be adequate for explaining BEF relationships]{Berlow2009,Wood2010,Saguin2014,Legagneux2014}, in combination with reproductive strategy and metabolism, are the basis for determining these groups.
\\\\
The \textsc{Madingley Model} has great potential for overcoming the limitations to BEF research posed by limited spatial and temporal scales. \\
In this light, and considering the role of predators highlighted earlier, this study aimed to test whether the presence of a particular predator group within a trophic level (i.e. group composition) has similarly decisive effects as previously shown, or if the number of groups present (i.e. functional richness) determines ecosystem functioning alone in complex ecosystems over large temporal and spatial scales. \\ 
For terrestrial systems, this represents the first attempt to elucidate - at such scales - the effect of functional diversity on ecosystem functioning.
\\\\
Here, strict carnivores were chosen as the focal group, of which the model defines three, two ectothermic (one iteroparous, one semelparous) and one endothermic iteroparous. 
\\
%Considering the fundamental differences between endotherms and ectotherms in terms of metabolic constraints \citep{Nagy2005}, it was hypothesized that (1) highly productive systems with constant climate would be dominated by ectotherms, as they have lower base metabolic requirements and are not constrained by temperature \citep{Buckley2012}; and (2) that the seasonal systems in temperate zones would, in contrast, pose productivity   and temperature constraints for ectotherms, promoting endotherm dominance as their thermoregulation allows them to remain active throughout unfavourable periods and increase their foraging potential \citep{Buckley2012}.\\\\
Considering the fundamental differences between endotherms and ectotherms in terms of metabolic constraints \citep{Nagy2005}, it was hypothesized that (1) highly productive systems with constant climate would be dominated by ectotherms (2) that seasonal systems in temperate zones would, in contrast, would promote endotherm dominance.


While for both ectotherms and endotherms metabolic activity increases with temperature \citep{Dillon2010}, requiring higher food intake, ecotherms have far lower base metabolic requirements, i.e. for maintenance, than endotherms of similar body mass \citep{Nagy1999}. Assuming an environment with constantly high (i.e. close to optimum) ambient temperatures, as well as abundant resources,  and lower energy expenditure for ectotherms, the latter can invest more energy into reproduction \cite{Shine2005}. Therefore, ectotherms are assumed to outperform endotherms.\\
In seasonal environments with periods of low-productiv and low temperatures, endotherms can maintain higher rates of activity for sustained periods of time \citep{Buckley2012}, as long as resource availability remains above a certain minimal threshold \cite{Wieser1985}. While ectotherms are known to undergo long periods of inactivity under unfavourable conditions \citep{Shine2005}, dormancy is not encoded into the model, and hence dynamics as hypothesized were expected to emerge. The simulated systems were chosen to reflect both temperature and productivity constraints, so that the outperfomance - with respect to the reasoning above - of either group would be notable.
\\\\
One equatorial, aseasonal and one temperate, seasonal ecosystem, chosen to identify potential interactions between biodiversity and climate, were simulated over 100 years for different combinations of carnivore groups at different levels of functional richness. The removal of a group was implemented so as to emulate instantaneous and complete extinction.\\
Ecosystem functioning, measured as primary production (i.e. autotroph biomass density), was recorded and related to community (abundance), and individual level (average body mass) properties across trophic groups.
%Considering the fundamental differences between endotherms and ectotherms in terms of metabolic constraints, it was hypothesized that (1) the equatorial, aseasonal system (highly productive) would be dominated by ectotherms, as they have lower base metabolic requirements and are not constrained by temperature; and (2) that the aseasonal system would pose productivity   and temperature constraints for ectotherms, promoting endotherm dominance.
%
%
%
%This is justified by recent findings suggesting that even simple organismal traits, such as body mass, are adequate for determining BEF relationships when combined with trophic identity \citep{Berlow2009,Wood2010,Saguin2014,Legagneux2014}, . 
%Including metabolic 
%
%%Perhaps most importantly, however, was the finding that maintaining multiple ecosystems simultaneously - as found in natural systems - required the presence of an even wider array of species \citep{Hector2007,Isbell2011, Cardinale2012}
%%This is linked to the different roles an organism can take on in an ecosystem given the specific traits it carries, and does allow grouping species into wider categories \citep{Hooper2002}.
%
%
%
%
%
%
%Effects of biodiversity loss were mainly inferred experimentally or via theoretical models by investigating diversity effects in random plant assemblages \citep[e.g.][]{Tilman2001,Hector2007,Isbell2011}, or between adjacent trophic levels \citep[mostly plant-herbivore interactions, e.g.][]{Thebault2003,Bruno2008} with limited species pools. \\
%Biodiversity loss was generally emulated by decreasing the number of randomly chosen species per assemblage or simulation \citep{Tilman2001,Thebault2003,Ives2005}, while trophic structure often served as a proxy of ecosystem functioning by measuring total biomass production and energy transfer between groups of organisms \citep[cf.][]{Tilman2014}.
%%However, extinction risks are higher for certain groups or taxa. such as apex predators \citep{Dobson2006}, or mammals \citep{Cardillo2005}, and experiments  considering non-random or odered extinction.
%Species richness was commonly used as a biodiversity metric, but later complimented by the introduction of functional diversity \citep[cf.][] to overcome limitations posed by rare or redundant species in a given ecosystem \citep[cf.][]{Hooper2002,Petchey2004}, generally explaining more variance in observed relationships \citep{Schmitz2015}. \cite{Duffy2007} illustrated the need of including higher degrees of trophic complexity into 


%multi-species interactions shape ecosystem functioning. 
%Several recent studies report multi-trophic effects cascading from predators to plants for select ecosystems, where increasing predator diversity indirectly enhanced primary producer performance in salt marshes \citep{Finke2004}, kelp forests \citep{Byrnes2006} and a number  of terrestrial ecosystems in western North America \citep{Beschta2009} by (experimentally) releasing herbivore pressure. 
%
%
%
%
%
%
%
%
%
%
% periodic climatic conditions used in the simulations and will most decidedly not be found in natural ecosystems, which are faced with variable environmental conditions and anthropogenic stressors. In fact, this heterogeneity in conditions is what often prevents generalisations from experiments to real-world systems \citep{Duffy2009, Hooper2012} and often even result in opposing effects when moving from small to large scales \citep{Mora2014}.  
%
%
%\textbf{OLD INTRO}
%Undoubtedly, one of the most unsettling truths of our time is that we cannot accurately predict how our actions will affect ecosystems and their functioning on regional, let alone planetary scale.\\
%Accelerating global change resulting from various human pressures, such as overexploitation, climate change and habitat destruction, is threatening the persistence of ecosystems and consequently human well-being \citep{MEA2005, Hooper2012}. In fact, these pressures, have already lead to a worrisome loss of biodiversity (\add{perhaps glossary?})
%with extinction rates rapidly increasing for various taxonomic groups \citep{Dirzo2003,Wake2008}.
%%in 2014 the IUCN (International Union for Conservation of Nature) listed 22413 of its 76199 assessed species as threatened by extintion \citep{IUCN2014}. 
%Predictions of biodiversity loss are equally grim: in a meta-analysis including results from 131 publications, \cite{Urban2015} estimated that climate change - as a singular driver - would threaten 16~\% of the planet's species with extinction under a business-as-usual scenario (average temperature increase of 4.3~$^{\circ}C$). 
%%This is especially disturbing as the current expert-consensus on reaching the international policy goal of limiting temperature increase to a 2~$^{\circ}C$ compared to pre-industrial conditions can no longer be achieved \citep{Fuss2014}.  
%\\\\
%%
% This \emph{extinction wave}, termed by \cite{Barnosky2011} as a potential sixth mass extinction event, is likely to impair ecosystem functioning and ecosystem service provisioning \citep{MEA2005,Naeem2012}, as
%%Biodiversity is a major determinant of ecosystem functioning, expressed through processes such as primary production, nutrient cycling, energy transfer between and within trophic levels, as well as responses to changing environmental conditions; 
%%recent research consistently suggests that 
%biodiversity loss decreases the rates at which essential processes (e.g. primary production, nutrient cycling, energy transfer between trophic levels) occur \citep[][]{Cardinale2012, Hooper2012}. \\
%Ecosystem functioning - or corresponding processes - have been shown to saturate with increasing biodiversity, potentially leading to a steep decrease or even cessation of an ecosystem process under continued extinction \citep{Cardinale2012,Tilman2014}. 
%Even though this pattern (concave-down) has generally been observed for simplified, experimental and agricultural systems, the concave-up relationship between biodiversity and ecosystem functioning found in natural systems could render effects of biodiversity loss even more detrimental \citep{Cardinale2012,Mora2014}. \\\\
%%
%Negative impacts may be further exacerbated, as functionally important species (\emph{keystone species}) are often more prone to extinction due to inherent traits, such as slow reproductive cycles and low population densities \citep{Cardillo2005}, selective pressures, such as hunting by humans \citep{Dobson2006}, or simply due to low abundances, restricted distributions or susceptibility to stochastic events \citep{MacArthur1967,Smith2003}. For example,
%% ungulates in the African savanna maintain the ecosystem's structure by controlling primary production, yet their replacement with functionally less diverse cattle has caused widespread degradation of these ecosystems \citep{DuToit1999}; 
%apex predators, such as wolves or lions, are often subject to selective hunting or extirpation \citep{Dobson2006,Ripple2014}, yet they can be of disproportionate importance for ecosystem functioning, when the top-down regulation they impose on a system's trophic structure is stronger than opposing bottom-up effects \citep{Borer2006,Estes2011}. 
% \\\\
% %
%Effects of biodiversity loss on biodiversity-ecosystem functioning (BEF) relationships were mainly inferred experimentally or via theoretical models by investigating diversity effects in random plant assemblages \citep[e.g.][]{Tilman2001,Hector2007,Isbell2011}, or between adjacent trophic levels \citep[mostly plant-herbivore interactions, e.g.][]{Thebault2003,Bruno2008} with limited species pools, commonly using species richness as a biodiversity metric. Theory underlying this research attributed biodiversity effects on ecosystem functioning to two mechanisms: 
%(1) complementarity (i.e. niche partitioning and facilitation), as the spectrum of available resources is more fully exploited while more species with differing strategies or requirements establish,
% and (2) selection effects, where single or few highly productive species dominate communities due to the prevalence of certain traits \citep{Loreau2001,Tilman2001}.\\
%Biodiversity loss was generally emulated by decreasing the number of randomly chosen species per assemblage or simulation \citep{Tilman2001,Thebault2003,Ives2005}, while trophic structuring often served as a proxy of ecosystem functioning by measuring total biomass production and energy transfer between groups of organisms \citep[cf.][]{Tilman2014}.
%  \add{species removal experiments, observational studies of BD loss}. \\\\
%%
%Some early research focused on ecosystem functioning in more complex systems, such as an observational study by \cite{Estes1974} on overgrazing of kelp forests by sea urchins after sea otter abundance had been greatly reduced, or work by \cite{Sih1998} elucidating how multi-species interactions shape ecosystem functioning. 
%Several more recent studies report multi-trophic effects cascading from predators to plants for select ecosystems, where increasing predator diversity indirectly enhanced primary producer performance in salt marshes \citep{Finke2004}, kelp forests \citep{Byrnes2006} and a number  of terrestrial ecosystems in western North America \citep{Beschta2009} by (experimentally) releasing herbivore pressure. 
%\add{multi-species predator-prey interactions \citep[cf.][]{Ives2005}}. 
%
%%
%However, species may be functionally redundant regarding an ecosystem process \citep{Hooper2002}, which can potentially mask diversity effects on ecosystem functioning when using species richness as the biodiversity metric \citep{Mora2014}.
% For example, \cite{OConnor2008} found that predator identity, rather than predator richness better determined variability of primary production in an estuarine community, hinting at functional redundancy within the system.\\ 
% This realization, along with other difficulties that can arise in large-scale experimental or observational studies in natural systems \citep[e.g. misidentification or low encounter probabilities during sampling;][]{Martinez1999}, spurred the development and application of functional diversity as a metric for studying BEF relationships.\\\\
% %
% Functional diversity (FD) is a measure of biodiversity based on traits (e.g. photosynthetic pathway, nutrient capture, foraging strategy) and/or trophic identity, akin to the niche and guild concept \citep{Hooper2002}.  
%This allows incorporating common, yet important, processes, such as intra-guild predation and omnivory \citep{Holt1997}, which can considerably alter the observed effects of functional diversity loss \citep{Ives2005}, without having to identify each agent taxonomically. Organisms can hence be attributed to distinct functional groups based on inherent characteristics, as well as in combination with their expression of traits in a continuous trait space (i.e. size, mass etc.).\\
%Recent findings \citep{Loeuille2005,Berlow2009,Wood2010,Saguin2014} suggest that even simple life-history and organismal traits, such as body mass and size, are adequate for determining BEF relationships when combined with trophic identity. 
%%\cite{Woodward2005} summarize the implications of body size on different ecosystem processes, such as nutrient cycling or secondary production.  \\ 
%\add{Examples of Functional Diversity Research and findings, add \citep{Wu2015} for recent multi-trophic study}
%%
%%Indeed, over the last decades, considerable effort has gone into BEF research investigating single and multi-trophic systems using  functional diversity, a metric based on traits (e.g. photosynthetic pathway, foraging strategy) and trophic identity, akin to the niche and guild concept \citep{Hooper2002}. 
%\\\\
%%
%%Species richness is not easily determined in real-world systems, as establishing exhaustive inventories is nigh impossible - species can be misidentified or may have a low encounter probability during sampling; yet rare species can have an important role for ecosystem functioning \citep{Jain2013}. Further, species are often functionally redundant, theoretically making a trait-based approach for measuring biodiversity-ecosystem functioning relationships more feasible  \\\\
%%More recently, attention has shifted towards identifying effects of horizontal (within a trophic level) and vertical (length of trophic chain) diversity \citep{Duffy2007}  on ecosystem functioning, acknowledging that intra-guild predation and omnivory govern the strength and can alter the sign of observed diversity effects \citep{Ives2005}. 
%
%
%%
%\textbf{Non-Random extinction}\\
%As outlined above, humanly driven extinctions can be non-random (e.g. due to selective hunting), and may have strong effects on ecosystem functioning when redundancies are few or exhausted \citep{ Wardle2011,Cardinale2012}. 
%While a number of empirical \citep[e.g.][]{Tilman2001,OConnor2008} and simulation-based \citep[e.g.][]{Borrvall2000,Ives2005} studies considered the effect of random loss of functional groups,  fewer have investigated the consequences of sequential or non-random loss of functional groups or decrementing of a continuous trait,  closer to extinction patterns observed in nature \citep{Solan2004, Raffaelli2004}. 
% Based on the assumption that larger organisms are more prone to extinction or extirpation \citep[e.g. after][]{Solan2004,Borer2006,Olden2007}, \cite{Saguin2014} showed that the effects of sequential extinction (i.e. ordered by body size) in multi-trophic marine and freshwater mesocosms were more detrimental to ecosystem functioning than random removal of taxa; further, they also identified body size in concert with trophic identity to be an easily measured indicator of ecosystem functioning. 
%%
%%decreased predator diversity results in higher abundances of herbivores (and potentially omnivores) increasing pressure on primary producers, while intraguild predation and omnivory dampen these cascading effects \citep{Duffy2007}. \add{non-random species citations: \cite{Fox2012,Wardle2011,Saguin2014}}
% \\\\
%%Trophic structuring often served as a proxy of ecosystem functioning by measuring total biomass production and energy transfer between groups of organisms. 
%%
%\textbf{Shortcomings of experimental studies}\\
%However, by design, these studies are restricted to simple experimental or observational settings, few interacting agents (i.e. species or functional groups) and are constrained in space and time \citep{Dobson2006,Estes2011,Cardinale2012}; this stands true even for comparatively long-lived experiments, such as the 20 year Cedar Creek experiment in Minnesota, USA \citep[][]{Tilman2001} or large-scale experiments, such as the IGRE species removal experiment in Mongolia, China \citep{Wu2015}.\\
%While these approaches guarantee tractability and often generate mechanistic understanding, their results can hardly be extrapolated to regional or global scales \citep{Bulling2006,Naeem2012};
% they further lack the ability to capture transient effects of e.g highly mobile agents that connect ecosystems functionally \citep{France2006}. \\
%%\rephrase{structure!}
%%and their ability to address questions such as the impact of non-random species loss from selective pressures on BEF in real-world or large-scale ecosystems \citep{Saguin2014,Wardle2011},
%Generating a comprehensive understanding of BEF relationships and how ecosystems are functionally linked would require investing in extensive research effort for areas that have already been studied partly, and even more so for remote areas that remain untouched. Unfortunately, ecosystems in these understudied regions (e.g. tropical rainforests in South America and Africa) are at highest risk of biodiversity loss, often due to their remoteness, yet carry disproportionally high numbers of endemic species \citep{Urban2015,Wake2008}.
%\\\\
%%Evidently, inferring BEF relationships from natural ecosystems (opposed to simple experimental or agricultural systems)  via experimental manipulation - and extrapolating to regional or global scales - may be impossible, as reaching adequate temporal and spatial scales to account for environmental variability is infeasible \citep{Duffy2007}. \\\\
%%
%%For example, an observational study by \cite{Estes1974} reported overgrazing of kelp forests by sea urchins after sea otter abundance had been greatly reduced. 
%%Multi-trophic effects (cascades) from predators to plants were found in several ecosystem types, where increasing predator diversity indirectly enhanced primary producer performance in salt marshes \citep{Finke2004}, kelp forests \citep{Byrnes2006} and a number  of terrestrial ecosystems in western North America \citep{Beschta2009} by (experimentally) releasing herbivore pressure. \add{multi-species predator-prey interactions \citep{Ives2005}}.\\\\
%%
%\textbf{Use of Models}\\
%Predicting the impact of biodiversity loss urgently requires mechanistic understanding of biodiversity-ecosystem functioning (BEF) relationships applicable to 'real-world' systems across the planet \citep{Dobson2006,Estes2011}. \add{\cite{Naeem2012}: identifies need for models}
%%
%
%%\section{Biodiversity and Ecosystem functioning}
%%\label{chap:intro:biodiv}
%%Here we go  here wo changes