\chapter{Discussion}
\label{chap:dis}
%
%The main objective of this study was to test the hypothesis that carnivore functional group composition was a stronger determinant of ecosystem functioning than functional richness \textit{per se}. Even though tested in various studies in the realm of biodiversity-ecosystem research (cf. following sections), to date there has been no attempt to extend this research to larger spatial and temporal scales in a trophically complex system, as is the case here.
%\add{Conclusion? General theme: predator identity affects ecosystem functioning} 
%\\\\
\section{Functional composition in high-productivity systems}
%\subsection{ESF}
Carnivore functional group composition, was a clear determinant of ecosystem functioning (expressed as autotroph biomass density) in the aseasonal system. 
Here, the presence of ectothermic, iteroparous carnivores ($Ect_i$), above all, was decisive for controlling herbivore pressure on autotrophs, clearly agreeing with hypothesised dynamics. 
This finding is in concert with results from multiple small-scale and mesocosm experiments \citep[e.g.][]{Finke2005, OConnor2008, Sanders2011}, where strong predator identity (i.e. composition) effects were found for multi-trophic species assemblages.\\\\
However, these effects generally have been  linked to intra-guild predation \citep{Ives2005}, differences between  generalist and specialist predators \citep{Duffy2007, Sanders2011}, as well as behaviourally-mediated trophic cascades \citep{Beschta2009, Schmitz2015}. 
The latter two are not formally encoded in the model, yet evidence presented here suggests that patterns qualitatively similar to empirical observations can emerge from the simple implementation of functional groups chosen here. 
%In the model, the degree of intra-guild predation in carnivores is governed by the body size spectrum that they realize following size-structured predation \citep{Williams2010}. 
%\\\\
%This is in agreement with work of \cite{Finke2005}, who found that %that ecosystem functioning (expressed as aboveground biomass of a focal primary producer species) was regulated not only by the number of predator species present, but by their function in the system. That is, 
%systems with single carnivore species can express the same functioning as those with multiple species present, as long as determining roles remain filled in the respective system. 
%Similarly,  identified the presence of single carnivore species as decisive for the trajectory of a system in a mesocosm experiment. While these results are specific to one species assemblage and a given ecosystem (e.g. mid-Atlantic coast of North America), they do give a strong implication for the effects of presence or absence of a functional group. The short study periods ensured that the studied systems hence not experience seasonality effects (i.e. productivity constraints), which makes mechanistic comparisons to an equatorial system system more viable.  \\
\subsection{Metabolic trade-offs and community composition}
The strong effect of ectothermic, iteroparous carnivore ($Ect_i$) presence on autotroph biomass density is assumed to be related to the constantly high ambient temperatures expected for an equatorial system in concert with abundant resources (i.e. climate does not constrain resource availability): for ectotherms, activity increases with temperature, while metabolic maintenance costs (i.e. field and basal metabolic rates) are  far lower than for endotherms of similar size \citep{Nagy2005,Buckley2012}. 
Hence, in the simulated aseasonal system, ectotherms can allocate more resources to reproduction than endothermic organisms. 
Given the large size range $Ect_i$ can realize in the model,  adequate control of herbivore pressure on autotrophs is reached across all size classes, maintaining ecosystem functioning similar to the control system. 
This reasoning closely follows the trophic release hypothesis by \cite{Hairston1960}. 
\\\\
In contrast, base metabolic costs for endotherms ($End_i$) are higher by at least an order of magnitude \citep{Nagy2005}. 
Hence, less resources can be allocated to reproduction, and fewer offspring is produced. 
As productivity is not a constrain, large $End_i$ are supported in the system \citep{Smith2011}, evident in high average body mass in \textit{exp. 7}, which are in turn longer-lived \citep{Speakman2005}. 
Assuming size-structred predation \citep{Williams2010}, small herbivores can escape predation, as only relatively few and large $End_i$ persist. \\\\
This reasoning can explain patterns found in \textit{exp. 5} and \textit{exp. 7}, where only $Ect_i$ and $End_i$ are present, respectively, but also in conjunction with the presence of ectothermic, semelparous organisms  $Ect_s$, which are constrained by a maximum body mass of 2~$kg$, and therefore fail to control larger-sized herbivores. For \textit{exp. 3}, where both $End_i$ and $Ect_i$ are present, competition,intraguild predation, as well suppression of large herbivores (given the size range that both carnivore groups can realize) are assumed to lead to smaller herbivores, at higher abundances. The control exerted by both carnivore groups, however, is still high enough to keep the system at control levels.
%
\section{Seasonality constraints on ecosystem functioning}
Ecosystem functioning for the seasonal system was predicted to remain constant, regardless of functional group composition and richness. Further, the expectation of endotherm dominance due to physiological advantages (i.e. thermoregulation) was not met, seeing that endotherms, in fact, died-off shortly after simulations began. The lack of downward cascading effects suggest that either (i) climate takes precedence over carnivore group composition and richness in controlling ecosystem functioning, as the low productivity phase in winter poses similar constraints on all functional groups, or (ii) that there is a minimum threshold of carnivore biomass density required for effectively exerting top-down control.\\\\
In temperate ecosystems, herbivore pressure during winter, when resources are scarce, can limit the recruitment of woody plant species \citep{Ripple2014}.
Interestingly, both \cite{Casabon2007} and \citep{Beschta2009} report that the structure of forests subject to increased browsing after large predators were extirpated, only changed decades later, as the standing stock (i.e. trees) remained intact, while only sapling recruitment failed.
While this can eventually alter ecosystem structure dramatically \citep{Terborgh2001, Estes2011}, there are no estimates of how biomass is redistributed within trophic groups, and whether total biomass remains constant within a trophic level and overall, as was observed here. \\\\
Regarding predation thresholds, \cite{Ripple2012}, as well as \cite{Johnson2009} found that systems with low predator density had higher abundances of herbivores. However, the lack of predators was linked to other external pressures, e.g. land-use change or hunting, as found in many other systems \citep{Estes2011,Ripple2014}. \\
Such effects were not included in experiments here. Hence, the emergence of such thresholds seem unlikely in the model, especially considering that available autotroph and herbivore biomass remained constant across experiments for the duration of the simulation. \\\\
Additional support for the assumption of climatic control is given by  \cite{Legagneux2014}. They found that low primary productivity in high-latitude, seasonal systems can be more limiting for large herbivores (i.e. abundance) than the control that predators impose from the top, as they are more likely to escape predation. 
This relates well to the little effect altered carnivore group composition and richness had on ecosystem functioning, as well as the large average body mass in $exp. 1$ (control). \\
They further identified summer temperatures as a governing factor of interaction strengths. Considering the approach used here, comparing maximum biomass densities during the most productive phase of a simulated year may have yielded more distinct patterns than using cross-year median values.
\subsection{Community and individual level dynamics}
A logical consequence of the near-constant biomass densities in the seasonal system is that any effects of changes in carnivore group composition or richness are reflected inx abundance and thereby average body mass, which is in order with predictions from by size-density relationships \citep[cf.][]{White2007}, and observed here. \\
Regarding carnivore functional groups, lower metabolic maintenance costs for ectotherms may give them an advantage over endotherms under resource scarce conditions in the low-productivity season \citep{Shine2005}, even though their activity may be restricted due to lower ambient temperatures. 
This could explain the early extinction of endothermic carnivores ($End_i$) in \textit{exp. 7}, which fail to meet their high metabolic requirements in during low productivity phases. Low initial cohort densities in this particular case may have exacerbated this effect by increased foraging and handling times, especially considering the constraint dispersal conditions (as only one grid cell was simulated).\\
Changes in size structure related to absence of ectothermic, iteroparous organisms ($Ect_s$) could be an artefact of considering average body masses, rather than detailed size distributions for all trophic or functional groups individually.
%
%\subsection{CIF}
%\add{Structure! Prominent example: Yellowstone trees, ecosystem structure, abundance of and numbers altered for ungulates, estimates of overall ecosystem functioning (i.e. overall biomass / carbon stocks for primary producers, perhaps only slight over all shift)}
%\\\\
%meta-analysis, \cite{Balvanera2006}: biodiversity effects weaker at ecosystem level, than community, matching predictions.  Over-representation of grassland / temperate systems, also matching system here.
%
%The cyclical patterns found in \textit{exp. 4, 6} and \textit{7} for carnivores may be an indication of dampened or lacking intra-guild interactions, where dynamics are governed by carnivores of similar body mass (i.e. roles), and thereby producing the stable, recurring pattern,  akin e.g. to a simple predator-prey model.


\section{Discrepancies with empirical estimates}
Empirical estimates of heterotroph-autotroph ratios for the terrestrial realm are generally sparse, as fully characterizing and quantifying a system is nearly impossible.\\ 
Perhaps unsurprisingly, the locations for the obtained values do not match the simulated ecosystems, and discrepancies are to be expected. \\
Next to this, the most likely reason for the markedly higher heterotroph-autotroph ratios is that the lack of dispersal in or out of the simulated systems may have lead to higher heterotroph abundances. The model \citep{Smith2012} underlying the autotroph plant stock is well tested against empirical data. Assuming that heterotroph dynamics, as opposed to false estimations of primary production, caused the marked shift, is therefore reasonable.\\ \cite{Harfoot2014} simulated systems at the same locations for 1000 years, and reported autotroph biomass densities between $10^{6}$ and $10^{8} kg\cdot km^{-2}$, more than an order of magnitude higher than the largest values here. 
In contrast to this study, they simulated four adjacent grid cells, perhaps leading to different size-density dynamics when dispersal in and out of a system is less limited.
\section{Implications for biodiversity and ecosystem service conservation }
Clearly, the \textsc{Madingley Model}, and approaches similar to the one applied here, cannot replace species-specific conservation  plans. Yet, it offers exciting opportunities , when one considers predicting how function and structure (i.e. individual sizes and abundances) of an ecosystem will respond to changing environmental conditions, e.g. in the context of global warming. This is especially useful when considering organisms that are physiologically sensitive to increasing temperatures.\\\\
Temperatures for the arctic are predicted to rise faster and higher than in the tropics  \citepalias{Parry2007} and yet impacts thereof are likely to be far more severe for ectotherms in lower latitudes: tropical ectotherms are already close to their temperature optimum, hence even small increases may have severe effects on their performance \citep{Deutsch2008}. \cite{Zeh2012} found that predicted temperature increases lead to drastic loss of sexual fitness and offspring survival rates for a tropical arachnid. Additionally considering that increased base metabolic rates require higher food intake \citep{Dillon2010}, ecosystem dynamics may be altered extremely. For such scenarios, the \textsc{Madingley Model} provides an ideal framework of predicting potential future trajectories of ecosystems, and  can inform management and policy alike.\\\\
The field of rewilding, generally in context of reintroducing large herbivores or predators, could greatly benefit from studies akin to this one, as a gap in understanding post-introduction dynamics is still prevalent in the literature \citep[e.g.][]{Grange2012,Smit2015}. Effects of altering the size structure of a trophic group can be investigated before reintroduction takes place. Outcomes on ecosystem level can be predicted, actions adjusted to management goals and supporting measures implemented if necessary - all before a single organism is displaced. \\\\ 
Ecosystem services are often linked to energy fluxes between and within trophic groups and nutrient cycles \citep{Cardinale2012}. As mentioned above, metabolic rates will increase with rising temperatures. Consequently, this also requires greater food intake. As herbivory can enhance nutrient cycling \citep{Belovsky2000}, this may affect systems in ways that general food web or population ecology would not predict. Being based on metabolic principles, the model can easily be adjusted to account for such dynamics. However, these often depend on behaviourally mediated interactions \cite{Hawlena2010}, and are therefore variable in nature, depending on which functional groups are present. This is not taken into account by the model currently, but offers a great opportunity for further exploration.
%\subsection{Biodiveristy-ccosystem functioning research}
%The degree of stability exhibited by both systems across experiments is a result of the constant,
%However, these can be easily implemented in the model framework, and will be the next step towards understanding how biodiversity and ecosystem functioning are linked, and what the response will be to future changes on local to global scales. 
%For future studies, the loss of a functional group could be simulated under scenarios of climate change and/or even be accompanied by the introduction of another group, resembling properties of an invasive species. 
% As functional group definitions are user-specified, and not fixed, there is great potential here.
\section{Model caveats}
The most obvious shortcoming of the model to-date is the lacking empirical basis for certain parameters. \\
In particular, activity times are set to constant values, while in fact, large ectotherms (i.e. reptiles) are known to have long periods of inactivity while digesting prey. 
The model does account for this, to some degree, by setting lower base metabolic requirements for ectotherms. However, the time spent foraging is most definitely over-estimated, which can potentially alter predicted ecosystem dynamics. 
In general, this holds true for all functional groups in which organisms may become dormant during low-productivity seasons. Predictions of lower-latitude cells with more constant climate are therefore more likely to be accurate. \\
Considering this, the combination of initial cohort densities, as well as high metabolic requirements during winter could have  caused the early extinction of endothermic carnivores in the seasonal system (\textit{exp. 7}) - a highly unexpected result. Investigating the size structure across trophic groups for this scenario in higher detail is likely to explain the observed outcome. A systematic error in the model's structure as the root cause could be excluded (oral communication with developer).
\\\\
As mentioned in previous sections, the model does not take into account behaviourally mediated trophic cascades. Yet these have been shown to be an important factor in determining predator-prey interactions \citep[e.g][]{Fortin2005,Duffy2007,Beschta2009}, and can thereby have a regulating effect on ecosystem functioning. This can lead to over or under-estimation of biomass fluxes, depending on the nature of such a cascade.\\ 
