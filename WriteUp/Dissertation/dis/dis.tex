\chapter{Discussion}
\label{chap:dis}
%
%The main objective of this study was to test the hypothesis that carnivore functional group composition was a stronger determinant of ecosystem functioning than functional richness \textit{per se}. Even though tested in various studies in the realm of biodiversity-ecosystem research (cf. following sections), to date there has been no attempt to extend this research to larger spatial and temporal scales in a trophically complex system, as is the case here.
%\add{Conclusion? General theme: predator identity affects ecosystem functioning} 
%\\\\
\section{Ecosystem functioning}
\subsection{Aseasonal system}
Carnivore functional group composition, was a clear determinant of ecosystem functioning (expressed as autotroph biomass density) in the aseasonal system. Here, the presence of ectothermic, iteroparous carnivores ($Ect_i$), above all, was decisive for controlling herbivore pressure on autotrophs. This finding is in agreement with results from multiple small-scale and mesocosm experiments \citep[e.g.][]{Finke2005, OConnor2008, Sanders2011} where similar effects were found for multi-trophic species assemblages.\\
However, these effects have been generally linked to intra-guild predation \citep{Ives2005}, differences between  generalist and specialist predators \citep{Duffy2007}, as well as behaviourally mediated trophic cascades \citep{Beschta2009, Schmitz2015}. 
The latter two are not formally encoded in the model, yet evidence presented here suggests that patterns qualitatively similar to empirical observations can emerge from the simple implementation of functional groups chosen here.
\\\\
The strong effect of $Ect_i$ presence on autotroph biomass density is assumed to be related to the constantly high ambient temperatures expected for an equatorial system in concert with abundant resources (i.e. productivity does not constrain resource availability): for ectotherms, activity increases with temperature, while metabolic maintenance costs (i.e. field and basal metabolic rates) are  far lower than for endotherms of similar size \citep{Nagy2005,Buckley2012}. 
Hence, in the simulated aseasonal system, ectotherms can allocate more resources to reproduction than endothermic organisms. 
Given the large size range $Ect_i$ can realize in the model,  adequate control of herbivore pressure on autotrophs is reached across all size classes, maintaining ecosystem functioning similar to the control system. 
This reasoning closely follows the trophic release hypothesis by \cite{Hairston1960}. 
\\\\
In contrast, base metabolic costs for endotherms ($End_i$) are at higher by at least an order of magnitude \citep{Nagy2005}. 
Hence, less resources can be allocated to reproduction, and fewer offspring is produced. 
As productivity is not a constrain, large $End_i$ are supported in the system \citep{Smith2011}, evident in high average body mass in $exp. 7$, which are in turn longer-lived \citep{Speakman2005}. 
Assuming size-structred predation \citep{Williams2010}, small herbivores could escape predation, as only relatively few and large $End_i$ persist. 
This reasoning can be applied to explain patterns found in \textit{exp. 5} and \textit{exp. 7}, where only $Ect_i$ and $End_i$ are present, respectively, but also in conjunction with the presence of ectothermic iteroparous organisms  $Ect_i$, which are constrained by a maximum body mass of 2~$kg$, and therefore fail to control larger-sized herbivores. 
%\\\\
%This is in agreement with work of \cite{Finke2005}, who found that %that ecosystem functioning (expressed as aboveground biomass of a focal primary producer species) was regulated not only by the number of predator species present, but by their function in the system. That is, 
%systems with single carnivore species can express the same functioning as those with multiple species present, as long as determining roles remain filled in the respective system. 
%Similarly,  identified the presence of single carnivore species as decisive for the trajectory of a system in a mesocosm experiment. While these results are specific to one species assemblage and a given ecosystem (e.g. mid-Atlantic coast of North America), they do give a strong implication for the effects of presence or absence of a functional group. The short study periods ensured that the studied systems hence not experience seasonality effects (i.e. productivity constraints), which makes mechanistic comparisons to an equatorial system system more viable.  \\
\subsection{Seasonal System}
Ecosystem functioning for the seasonal system was predicted to remain constant, regardless of functional group composition and richness. The lack of downward cascading effects suggest that either (i) climate takes precedence over carnivore group composition and richness in controlling ecosystem functioning  or (ii) that there is a minimum threshold of carnivore biomass density required for effectively exerting top-down control.

\section{Community and Individual dynamics}
\add{Structure! Prominent example: Yellowstone trees, ecosystem structure, abundance of and numbers altered for ungulates, estimates of overall ecosystem functioning (i.e. overall biomass / carbon stocks for primary producers, perhaps only slight over all shift)}
\\\\
meta-analysis, \cite{Balvanera2006}: biodiversity effects weaker at ecosystem level, than community, matching predictions.  Over-representation of grassland / temperate systems, also matching system here.

The cyclical patterns found in \textit{exp. 4, 6} and \textit{7} for carnivores may be an indication of dampened or lacking intra-guild interactions, where dynamics are governed by carnivores of similar body mass (i.e. roles), and thereby producing the stable, recurring pattern,  akin e.g. to a simple predator-prey model.

\section{Prediction caveats}
Role of behaviourally mediated trophic cascades,  as found in yellowstone (Ripple, Beschta);




 and for grasshoppers and spiders of different predation mode (stalking, sit and wait), can have large effect on element cycling \cite{Schmitz2015}.