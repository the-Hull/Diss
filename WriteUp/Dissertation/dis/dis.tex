\chapter{Discussion}
\label{chap:dis}
%
%The main objective of this study was to test the hypothesis that carnivore functional group composition was a stronger determinant of ecosystem functioning than functional richness \textit{per se}. Even though tested in various studies in the realm of biodiversity-ecosystem research (cf. following sections), to date there has been no attempt to extend this research to larger spatial and temporal scales in a trophically complex system, as is the case here.
%\add{Conclusion? General theme: predator identity affects ecosystem functioning} 
%\\\\
\section{Functional composition in high-productivity systems}
%\subsection{ESF}
Carnivore functional group composition, was a clear determinant of ecosystem functioning (expressed as autotroph biomass density) in the aseasonal system. 
Here, the presence of ectothermic, iteroparous carnivores ($Ect_i$), above all, was decisive for controlling herbivore pressure on autotrophs. 
This finding is in agreement with results from multiple small-scale and mesocosm experiments \citep[e.g.][]{Finke2005, OConnor2008, Sanders2011}, where strong predator identity (i.e. composotion), effects were found for multi-trophic species assemblages.\\
However, these effects generally have been  linked to intra-guild predation \citep{Ives2005}, differences between  generalist and specialist predators \citep{Duffy2007}, as well as behaviourally-mediated trophic cascades \citep{Beschta2009, Schmitz2015}. 
The latter two are not formally encoded in the model, yet evidence presented here suggests that patterns qualitatively similar to empirical observations can emerge from the simple implementation of functional groups chosen here. 
%In the model, the degree of intra-guild predation in carnivores is governed by the body size spectrum that they realize following size-structured predation \citep{Williams2010}. 
%\\\\
%This is in agreement with work of \cite{Finke2005}, who found that %that ecosystem functioning (expressed as aboveground biomass of a focal primary producer species) was regulated not only by the number of predator species present, but by their function in the system. That is, 
%systems with single carnivore species can express the same functioning as those with multiple species present, as long as determining roles remain filled in the respective system. 
%Similarly,  identified the presence of single carnivore species as decisive for the trajectory of a system in a mesocosm experiment. While these results are specific to one species assemblage and a given ecosystem (e.g. mid-Atlantic coast of North America), they do give a strong implication for the effects of presence or absence of a functional group. The short study periods ensured that the studied systems hence not experience seasonality effects (i.e. productivity constraints), which makes mechanistic comparisons to an equatorial system system more viable.  \\
\subsection{Metabolic trade-offs and community composition}
The strong effect of $Ect_i$ presence on autotroph biomass density is assumed to be related to the constantly high ambient temperatures expected for an equatorial system in concert with abundant resources (i.e. productivity does not constrain resource availability): for ectotherms, activity increases with temperature, while metabolic maintenance costs (i.e. field and basal metabolic rates) are  far lower than for endotherms of similar size \citep{Nagy2005,Buckley2012}. 
Hence, in the simulated aseasonal system, ectotherms can allocate more resources to reproduction than endothermic organisms. 
Given the large size range $Ect_i$ can realize in the model,  adequate control of herbivore pressure on autotrophs is reached across all size classes, maintaining ecosystem functioning similar to the control system. 
This reasoning closely follows the trophic release hypothesis by \cite{Hairston1960}. 
\\\\
In contrast, base metabolic costs for endotherms ($End_i$) are higher by at least an order of magnitude \citep{Nagy2005}. 
Hence, less resources can be allocated to reproduction, and fewer offspring is produced. 
As productivity is not a constrain, large $End_i$ are supported in the system \citep{Smith2011}, evident in high average body mass in \textit{exp. 7}, which are in turn longer-lived \citep{Speakman2005}. 
Assuming size-structred predation \citep{Williams2010}, small herbivores can escape predation, as only relatively few and large $End_i$ persist. 
This reasoning can be applied to explain patterns found in \textit{exp. 5} and \textit{exp. 7}, where only $Ect_i$ and $End_i$ are present, respectively, but also in conjunction with the presence of ectothermic iteroparous organisms  $Ect_i$, which are constrained by a maximum body mass of 2~$kg$, and therefore fail to control larger-sized herbivores. 




\section{Seasonality constraints on ecosystem functioning}
Ecosystem functioning for the seasonal system was predicted to remain constant, regardless of functional group composition and richness. The lack of downward cascading effects suggest that either (i) climate takes precedence over carnivore group composition and richness in controlling ecosystem functioning, as the low productivity phase in winter poses similar constraints on all functional groups, or (ii) that there is a minimum threshold of carnivore biomass density required for effectively exerting top-down control.\\\\
In temperate ecosystems, herbivore pressure during winter, when resources are limited, can limit the recruitment of woody plant species \citep{Ripple2014}.
 This effect is exacerbated when herbivore control by predation is decreased. 
While this can alter ecosystem structure dramatically \citep{Terborgh2001, Estes2011}, there are no estimates of how biomass is redistributed within trophic groups, and whether total biomass remains constant within a trophic level on ecosystem level. \\\\
Regarding predation thresholds, \cite{Ripple2012}, as well as \cite{Johnson2009} found that systems with low predator density had higher abundances of herbivores. However, the lack of predators was linked to other external pressures, e.g. land-use change or hunting, as found in many other systems \citep{Estes2011,Ripple2014}. Such effects were not included in experiments here. Hence, the emergence of such thresholds seems unlikely in the model, especially considering that available autotroph and herbivore biomass remained constant across experiments. In addition,  \cite{Legagneux2014} found that low primary productivity in high-latitude, seasonal systems can be more limiting for large herbivores (i.e. abundance) than the control that predators impose from the top, as larger herbivores are more likely to escape predation. 
This relates well to the little effect altered carnivore group composition and richness had on ecosystem functioning, as well as the large average body mass in $exp. 1$ (control). \\
They further identified summer temperatures as a governing factor of interaction strengths. Considering the approach used here, comparing maximum biomass densities during the most productive phase of a simulated year may yield more distinct patterns than using inter-annual median values.
\subsection{Community and Individual level dynamics}
A logical consequence of the near-constant biomass densities in the seasonal system is that any effects of changes in carnivore group composition or richness are reflected in either abundance and thereby average body mass. As predicted by size-density relationships \citep[cf.][]{White2007}, lower body mass coincides with higher abundances. \\
Regarding carnivore functional groups, lower metabolic maintenance costs for ectotherms may give them an advantage over endotherms under resource scarce conditions in the low-productivity season \citep{Shine2005}, even though their activity may be restricted due to lower ambient temperatures. 
This could explain the early extinction of endothermic carnivores ($End_i$) in \textit{exp. 7}, which fail to meet their high metabolic requirements in during low productivity phases. This may have been exacerbated by initial cohort densities being lower and thereby increasing foraging and handling times were, as well as limiting dispersal by only simulating individual grid cells.\\
Changes in size structure related to absence of ectothermic, iteroparous organisms ($Ect_s$) could be an artefact of considering average body masses, rather than detailed size distributions for all individual trophic or functional groups.
%
%\subsection{CIF}
%\add{Structure! Prominent example: Yellowstone trees, ecosystem structure, abundance of and numbers altered for ungulates, estimates of overall ecosystem functioning (i.e. overall biomass / carbon stocks for primary producers, perhaps only slight over all shift)}
%\\\\
%meta-analysis, \cite{Balvanera2006}: biodiversity effects weaker at ecosystem level, than community, matching predictions.  Over-representation of grassland / temperate systems, also matching system here.
%
%The cyclical patterns found in \textit{exp. 4, 6} and \textit{7} for carnivores may be an indication of dampened or lacking intra-guild interactions, where dynamics are governed by carnivores of similar body mass (i.e. roles), and thereby producing the stable, recurring pattern,  akin e.g. to a simple predator-prey model.

\section{Real-world implications}
%
The degree of stability exhibited by both systems across experiments is a result of the constant, periodic climatic conditions used in the simulations and will most decidedly not be found in natural ecosystems, which are faced with variable environmental conditions and anthropogenic stressors. 
\\\\
\subsection{Discrepancies with empirical estimates}
Empirical estimates of heterotroph-autotroph ratios for the terrestrial realm are generally sparse, as fully characterizing and quantifying a system is nearly impossible.\\ 
Perhaps unsurprisingly, the locations for the obtained values do not match the simulated ecosystems, and discrepancies are to be expected. Next to this, the most likely reason for the markedly higher heterotroph-autotroph ratios is that the lack of dispersal in or out of the simulated systems may have lead to higher heterotroph abundances. The model \citep{Smith2012} underlying the autotroph plant stock is well tested against empirical data. Assuming that heterotroph dynamics, in contrast to false estimations of primary production, caused the marked shift is therefore reasonable. \cite{Harfoot2014} simulated systems at the same locations for 1000 years, and reported autotroph biomass densities between $10^{6}$ and $10^{8} kg\cdot km^{-2}$, more than an order of magnitude higher than the largest values here. In contrast to this study, they simulated four adjacent grid cells, perhaps leading to different size-density dynamics.
\subsection{Conservation of biodiversity and ecosystem services}

\section{Model caveats and unmet expectations}
Role of behaviourally mediated trophic cascades,  as found in yellowstone (Ripple, Beschta);

 and for grasshoppers and spiders of different predation mode (stalking, sit and wait), can have large effect on element cycling \cite{Schmitz2015}.
 
No detailed available about size structure (only average values)