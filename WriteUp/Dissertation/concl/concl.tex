\chapter{Conclusion}
The degree of stability exhibited by both systems across experiments is a result of the constant, periodic climatic conditions used in the simulations and will most decidedly not be found in natural ecosystems, which are prone to variable environmental and anthropogenic stressors. However, the goal of this study was to investigate biodiversity effects on ecosystem functioning on larger temporal and spatial scales than attempted before, illustrating the potential of cross-ecosystem comparisons. In this light, a number of insights were generated, and potential new lines of research to explore identified.\\
The major finding here was that 
community level (abundance density) and individual (average body mass) properties govern the response of a trophic group on ecosystem level (biomass density). While this has been reported previously, this study could expand the results to ecosystem functioning on large scales with systems of high complexity. 
That is, the ecosystem level property considered here will only be affected when changes in average body mass coincide with an appropriate directional shift of overall abundance in the respective trophic group (e.g. heavier organisms at higher abundances mean higher body mass densities)