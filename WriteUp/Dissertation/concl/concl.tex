\chapter{Conclusion}
%The degree of stability exhibited by both systems across experiments is a result of the constant, periodic climatic conditions used in the simulations and will most decidedly not be found in natural ecosystems, which are prone to variable environmental and anthropogenic stressors. 
%The goal of this study was to investigate biodiversity effects (i.e. loss of a functional group) on ecosystem functioning on larger temporal and spatial scales than attempted before, while illustrating the potential of a general ecosystem model and cross-ecosystem comparisons. In this light, a number of insights were generated, and potential new lines of research to explore identified.\\\\
A need for studies elucidating the role of biodiversity for ecosystem functioning and structure in systems of natural complexity has been expressed repeatedly \citep[e.g.][]{Hooper2012,Naeem2012,Tilman2014}.\\
Here, a relationship between biodiversity and ecosystem functioning, that was previously identified in small-scale or short-term experiments, has been successfully reproduced using a general ecosystem model providing high degrees of complexity at adequate scale:
carnivore group composition was shown to govern individual (average body mass) and community level (abundance density)  properties and thereby the response of a trophic group on ecosystem level (biomass density) for an aseasonal, highly productive systems. \\\\
In addition, the potential existence of a climatically driven mechanism in seasonal environments, where functional group composition alters ecosystem structure, rather than functioning, 
%by redistributing biomass within trophic groups 
was highlighted and begs further investigation.\\
Including dispersal between adjacent grid cells, as well as tracking changes in size distributions within a functional group, offer great moments of motivation to elaborate on findings presented here. 
\\\\
\add{Conservation implications}.
 However, the most exciting feature of the \textsc{Madingley Model} may not be to test existing hypothesis, but to generate new questions. Or, even more so in the light of current rates of biodiversity loss \citep{Ceballos2015, Urban2015}, to answer them before they turn into grave problems for society and generations to come. 
