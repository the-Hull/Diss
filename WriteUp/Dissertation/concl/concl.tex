\chapter{Conclusion}
%The degree of stability exhibited by both systems across experiments is a result of the constant, periodic climatic conditions used in the simulations and will most decidedly not be found in natural ecosystems, which are prone to variable environmental and anthropogenic stressors. 
%The goal of this study was to investigate biodiversity effects (i.e. loss of a functional group) on ecosystem functioning on larger temporal and spatial scales than attempted before, while illustrating the potential of a general ecosystem model and cross-ecosystem comparisons. In this light, a number of insights were generated, and potential new lines of research to explore identified.\\\\
This study successfully reproduced previously identified biodiversity-ecosystem functioning relationships - at larger spatial and temporal scales than attempted before - using a general ecosystem model:
Carnivore group composition governed both community level (abundance density) and individual (average body mass) properties thereby the response of a trophic group on ecosystem level (biomass density) for an aseasonal, highly productive systems. \\ 
The potential existence of a climatically driven mechanism, where functional group composition alters ecosystem structure, rather than functioning, by redistributing biomass within trophic groups was highlighted and begs further research.
Further, including dispersal between adjacent grid cells, as well as tracking changes in size distributions within a functional group, is likely to unravel more intricate relationships between functional diversity and ecosystem functioning.\\\\
\add{Conservation implications} 