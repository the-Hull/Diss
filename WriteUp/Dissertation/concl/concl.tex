\chapter{Conclusion}
%The degree of stability exhibited by both systems across experiments is a result of the constant, periodic climatic conditions used in the simulations and will most decidedly not be found in natural ecosystems, which are prone to variable environmental and anthropogenic stressors. 
%The goal of this study was to investigate biodiversity effects (i.e. loss of a functional group) on ecosystem functioning on larger temporal and spatial scales than attempted before, while illustrating the potential of a general ecosystem model and cross-ecosystem comparisons. In this light, a number of insights were generated, and potential new lines of research to explore identified.\\\\
A need for studies elucidating the role of biodiversity for ecosystem functioning and structure in systems of natural complexity has been expressed repeatedly \citep[e.g.][]{Hooper2012,Naeem2012,Tilman2014}.\\\\
Here, a relationship between biodiversity and ecosystem functioning, that was previously identified in small-scale or short-term experiments, has been successfully reproduced using a general ecosystem model providing high degrees of complexity at adequate scale:
functional group composition, more precisely the presence of ectothermic, iteroparous carnivores, was shown to govern individual and community level  properties of herbivores, and thereby effected an ecosystem level response (biomass density) for autotrophs in a highly productive, aseasonal system. While functional richness (number of groups present) also had an effect, the dominance of a single functional group for governing ecosystem functioning was striking, albeit hypothesised.  \\

Ecosystem functioning remaind virtually constant for the seasonal system, while community (abundance) and individual level (average body mass) properties shifted markedly across trophic groups and experiments. This implied the potential existence of a climatically driven mechanism in seasonal environments, where functional group composition alters ecosystem structure, rather than functioning, which most decidedly excites further research.
%by redistributing biomass within trophic groups 
 \\
Including dispersal between adjacent grid cells, as well as tracking changes in size distributions within a functional group, are likely to refine conclusions drawn here and give further insight to mechanisms driving the observed changes. 
\\\\
The results indicate that previously inferred mechanisms may only apply to subsets of natural systems. Research must be extended to include a variety of environments and generate more comprehensive understanding of biodiversity-ecosystem functioning relationships, if management of ecosystems is to be successful in the face of climate change and increasing rates of biodiversity loss. 
\\
Nonetheless,  potential applications for the conservation of ecosystem services, and especially management actions that imply altering a systems size structure (e.g. in the context of rewilding with large herbivores or predators) seem most suitable for being answered within a similar framework as this study using the Madingley Model.
\\\\
%In conclusion, the most stimulating feature of the \textsc{Madingley Model} may not be to test existing hypotheses, but to generate new questions that can be answered before they turn into grave problems for society. In the context of this study, investigating at which temperature the dynamics observed   
